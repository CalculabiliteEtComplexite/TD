% SPDX-License-Identifier: CC-BY-SA-4.0
% Author: Matthieu Perrin
% Part: Languages
% Section: Chomsky hierarchy
% Exercise: Identification of grammar types

\begingroup

\begin{exercice}[Types de grammaires et de langages]\label{exo:languages/chomsky/types}

  Pour chacune des grammaires suivantes, répondre aux questions suivantes :

  \begin{enumerate}
  \item Donner deux mots appartenant au langage, avec pour chacun une dérivation complète,
    ainsi que deux mots n'appartenant pas au langage.
  \item Indiquer à quels types de la hiérarchie de Chomsky la \emph{grammaire} appartient.
  \item Décrire, sous forme ensembliste, le \emph{langage} engendré par la grammaire.
  \item Le langage admet-il une grammaire d'un type \emph{plus restrictif} que celle donnée ?
  \end{enumerate}

  \begin{correction}
    Le type 0 est toujours facile à vérifier : il suffit de s'assurer que c'est bien une grammaire
    (bon nombre d'éléments dans le tuple, bon types d'éléments, et au moins un non-terminal à gauche de chaque règle).
    En général, ce sera toujours le cas.

    Ensuite, le type 2 est très simple à vérifier également :
    il faut vérifier que le membre gauche de chaque règle est un et un seul non-terminal.

    Si la grammaire n'est pas de type 2, elle n'est pas de type 3.
    Sinon, on doit vérifier qu'elle est linéaire à gauche ou à droite,
    et que chaque membre droit contient au plus un terminal. 

    Si la grammaire est de type 2, il suffit de vérifier les membres droits vides pour vérifier si elle est de type 1.
    Sinon, il faut identifier les contextes gauches et droits de chaque règle. 
  \end{correction}

  \begin{question}
  \item $G_1 \eqdef \left\langle \{a, b\}, \{S, T\}, S,
    \left\{\begin{array}{rcl}
    S &\rightarrow& a T \mid b\\
    T &\rightarrow& S a
    \end{array}\right\}  \right\rangle$
  \end{question}
  \begin{correction}
    \begin{enumerate}
    \item Mots dans le langage :
      $S \vdash b$ et
      $S \vdash aT \vdash aSa \vdash ab a$.\\
      Mots \emph{pas} dans le langage : $a$, $ab$, $baa$, etc.
    \item
      Les côtés gauches sont d'un seul non-terminal, donc grammaire de type~2.  
      La règle $S \to aT$ est régulière à droite, et la règle $T \to Sa$ est régulière à gauche.
      Les grammaires de type 3 n'autorisent pas à malanger les deux, donc pas de type~3.
      Il n'y a pas de $\varepsilon$ à droite des règles, donc type $1$. 
      La grammaire est donc de types 2,1,0.
    \item $L(G_1) = \{ a^n b a^n \mid n \ge 0 \}$.
    \item
      $L(G_1)$ n'est pas rationnel (lemme de l'étoile) donc il n'existe pas de grammaire de type~3 pour ce langage.
      Le langage est donc de type $2$. 
    \end{enumerate}
  \end{correction}

  \begin{question}
  \item $G_2 \eqdef \left\langle \{a, b\}, \{S, T\}, S,
    \left\{\begin{array}{rcl}
    S &\rightarrow& a S \mid T\\
    a T &\rightarrow& a T a \\
    T &\rightarrow& b
    \end{array}\right\}  \right\rangle$
  \end{question}
  \begin{correction}
    \begin{enumerate}
    \item Mots dans le langage :
      $S \vdash T \vdash b$ et 
      $S \vdash aS \vdash aT \vdash aTa \vdash ba a$.
      Mots hors du langage : $\varepsilon$, $a$, $bb$, $ab$, etc.  
    \item
      La règle $aT \to aTa$ a un côté gauche de longueur $2$ : la grammaire n'est pas de type~2, donc pas de type~3.  
      La grammaire est bien de type 1, et donc aussi 0. 
    \item $L(G_2) = \{\, a^p b a^q \mid p,q \ge 0 \,\} = a^\star b a^\star$.
    \item $L(G_2)$ est rationnel (expression régulière $a^\star b a^\star$), donc il existe une grammaire de type~3
      pour ce langage (par exemple $S \to aS \mid bA$, $A \to aA \mid \varepsilon$).  
    \end{enumerate}
  \end{correction}

  \begin{question}
  \item $G_3 \eqdef \left\langle \{a, b, c\}, \{S, A, B, C\}, S,
    \left\{\begin{array}{rcl}
    S   &\rightarrow& \varepsilon \mid  A \\
    A   &\rightarrow& aA \mid  B \\
    B   &\rightarrow& bB \mid  C \\
    C   &\rightarrow& cC \mid  c\\
    \end{array}\right\}  \right\rangle$
  \end{question}
  \begin{correction}
    \begin{enumerate}
    \item Mots dans le langage :
      $S \vdash A \vdash aA \vdash B \vdash bB \vdash C \vdash c$ et
      $S \vdash A \vdash aA \vdash aA \vdash B \vdash C \vdash cC \vdash cc$
      Mots hors du langage : $a$, $b$, $ac$, $bc$, $c^0=\varepsilon$ avec des $b$ ou $c$ mal placés, etc.
    \item Toutes les productions sont de la forme $X \to aX$, $X \to Y$, ou $X \to a$,
      avec un unique non-terminal en tête, éventuellement remplacé par un terminal
      ou un terminal suivi d'un non-terminal en fin de mot, plus $S \to \varepsilon$.  
      C'est une grammaire linéaire droite, donc de type~3 (et donc aussi de types 2,1,0).
    \item $L(G_3) = \varepsilon \mid a^\star b^\star c^+$.
    \item Le langage est rationnel, donc de type minimal 3.  
    \end{enumerate}
  \end{correction}

  \begin{question}
  \item $G_4 \eqdef \left\langle \{a, b\}, \{S\}, S,
    \left\{\begin{array}{rcl}
    S   &\rightarrow& \varepsilon \mid  aSaS\mid  bSaS\\
    SaS &\rightarrow& SbS 
    \end{array}\right\}  \right\rangle$
  \end{question}
  \begin{correction}
    \begin{enumerate}
    \item Mots dans le langage :
      $S \vdash \varepsilon$,
      $S \vdash aSaS \vdash a S a S \vdash a a S \vdash aa$ et
      $S \vdash aSaS \vdash aSaS aS \vdash aSbS aS \vdash abS aS \vdash ab a S \vdash abaS \vdash ab a b S \vdash abab$.
      Mots hors du langage : $a$, $b$, $aba$, $abb$, etc.  
    \item La règle $SaS \to SbS$ a un côté gauche de longueur $3$,
      donc la grammaire n'est pas de type~2.  
      De plus, la règle $S \to \varepsilon$ est contractante et $S$ apparaît à
      droite d'autres règles, donc la grammaire n'est pas de type~1.  
      La grammaire est de type 0 uniquement.
    \item $L(G_4) = \left((a \mid b)^2\right)^\star$.
      
      Intuition : chaque application de $S \to aSaS$ ou $S \to bSaS$ ajoute deux lettres
      terminales après que les deux $S$ restants auront dérivé en mots terminaux,
      et la règle $SaS \to SbS$ ne change pas la longueur du mot (elle transforme un $a$ en $b$ en position centrale).  
      On en déduit que tout mot dérivé a une longueur paire.  
      Réciproquement, on peut construire par récurrence tout mot de longueur paire en choisissant à chaque
      étape $aSaS$ ou $bSaS$ puis en remplaçant certains $SaS$ par $SbS$ pour ajuster les lettres.
    \item Le langage est rationnel, donc il admet donc une grammaire de type~3.  
    \end{enumerate}
  \end{correction}

\end{exercice}

\endgroup
\endinput
