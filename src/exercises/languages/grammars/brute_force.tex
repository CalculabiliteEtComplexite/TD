% SPDX-License-Identifier: CC-BY-SA-4.0
% Author: Matthieu Perrin
% Part: Languages
% Section: Contextual languages
% Exercise: Brute force ascending algorithm

\begingroup

\begin{exercice}[Recherche ascendante par force brute]\label{exo:/languages/grammars/brute_force}

  Soit la grammaire contextuelle suivante :

  $$G \eqdef \left\langle\{a, b, c\}, \{S, A, B, C\}, S,
  \left\{\begin{array}{lll}
  S   &\rightarrow& Abc \mid  ABSc\\
  BA  &\rightarrow& CA\\
  CA  &\rightarrow& CB\\
  CB  &\rightarrow& AB\\
  A  &\rightarrow& a\\
  Bb  &\rightarrow& bb\\
  \end{array}\right\} \right\rangle$$
  
  \begin{question}
  \item Trouver 2 mots appartenant au langage et 2 mots n'appartenant pas au langage.
  \end{question}
  \begin{correction}
    Des exemples de
    mots du langage~: $abc$, $aabbcc$, $aaabbbccc$, etc. Des exemples de mots
    qui ne sont pas dans le langage (pour des raisons évidentes)~: tous les mots sans $c$
    ou sans le facteur $ab$. 
  \end{correction}
  
  \begin{question}
  \item Utiliser l'algorithme de recherche ascendante par force brute pour
    déterminer si $cb$, $abbc$, $abc$ et $aabbcc$ appartiennent au langage.
  \end{question}
  \begin{correction}
    Pour la recherche ascendante par force brute
    (dans les réécritures inverses $\leftarrow$ 
    on ne garde que les nouveaux mots)~:
    \begin{itemize}
    \item $\{cb\} \leftarrow \emptyset \Rightarrow$ $cb$ n'appartient pas au langage.
    \item $\{abbc\} \leftarrow \{Abbc, aBbc\} \leftarrow
      \{ABbc\} \leftarrow \{CBbc\} \leftarrow \{CAbc\}
      \leftarrow \{BAbc, CS\} \leftarrow \{BS\} \leftarrow \emptyset
      \Rightarrow$ $abbc$ n'appartient pas au langage.
    \item $\{abc\} \leftarrow \{Abc, S\} \Rightarrow$ $abc$ appartient langage.
    \item Très long. On peut donner une dérivation (inverse) du mot~:
      $aabbcc \leftarrow Aabbcc \leftarrow AAbbcc \leftarrow AABbcc 
      \leftarrow ACBbcc \leftarrow ACAbcc \leftarrow ABAbcc \leftarrow ABSc \leftarrow S\Rightarrow$  $aabbcc$ appartient au langage.
    \end{itemize}
  \end{correction}

  \begin{question}
  \item Déterminer le langage associé à la grammaire.
  \end{question}
  \begin{correction}
    Le langage est $\{a^nb^nc^n \mid  n > 0\}$.
  \end{correction}
  
\end{exercice}

\endgroup
\endinput
