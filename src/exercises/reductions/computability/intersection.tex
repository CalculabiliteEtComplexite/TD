% SPDX-License-Identifier: CC-BY-SA-4.0
% Author: Matthieu Perrin
% Part: Introduction
% Section: Words and languages
% Exercise: Words

\begingroup

\begin{exercice}[Intersection de langages algébriques]\label{exo:reductions/computability/intersection}
  Soit $\Sigma = \{a, b\}$ un alphabet. On rappelle la définition du Problème de Correspondance de Post (PCP) sur $\Sigma$.

  \Probleme{PCP}{
    Un ensemble fini non-vide de dominos $D = \left\{ \VDomino{u_1}{v_1}, ..., \VDomino{u_n}{v_n} \right\} \subseteq \Sigma^+ \times \Sigma^+$.
  }{
    Existe-t-il un mot $\VDomino{u_{i_1}}{v_{i_1}} \cdots \VDomino{u_{i_k}}{v_{i_k}} \in D^+$ tel que $u_{i_1} \cdots u_{i_k} = v_{i_1} \cdots v_{i_k}$ ?
  }

  \begin{question}
  \item\label{ex2.Q1} Les instances $D_1$ et $D_2$ suivantes de PCP sont-elles positives ? Justifiez en donnant une solution ou une preuve qu'il n'en existe pas pour chaque instance.
    $$D_1 \eqdef \left\{
    \VDomino[8mm]{aba}{a},
    \VDomino[8mm]{a}{ababa},
    \VDomino[8mm]{ab}{b},
    \VDomino[8mm]{b}{ab}
    \right\}
    \quad\quad\quad
    D_2 \eqdef \left\{
    \VDomino[8mm]{$aba$}{$ba$},
    \VDomino[8mm]{$aa$}{$a$},
    \VDomino[8mm]{$b$}{$bb$}
    \right\}$$
  \end{question}
  \begin{correction}
    \begin{itemize}
    \item $D_1$ est positive : 
      $ \VDomino[8mm]{a}{ababa}\cdot 
      \VDomino[8mm]{b}{ab}\cdot
      \VDomino[8mm]{aba}{a}\cdot
      \VDomino[8mm]{aba}{a}\cdot
      \VDomino[8mm]{ab}{b}$ est une solution. 
    \item $D_2$ est négative : une solution ne peut pas commencer par \VDomino[8mm]{aba}{ba} car les deux mots ne commencent pas par la même lettre ;
      si on commence par \VDomino[8mm]{aa}{a}, on est forcé de répéter \VDomino[8mm]{aa}{a} sans jamais atteindre l'égalité ;
      idem pour \VDomino[8mm]{b}{bb}. 
    \end{itemize}
  \end{correction}
  
  On définit maintenant le problème de l'intersection des langages algébriques :
  \Probleme{Intersect-Alg}{
    Une paire $I = \langle G, H \rangle$ de grammaires algébriques.
  }{
    Existe-t-il un mot $w \in \mathcal{L}(G) \cap \mathcal{L}(H)$ ?
  }

  \begin{question}
  \item Les instances $I_1$ et $I_2$ suivantes de \textsc{Intersect-Alg} sont-elles positives ?
    Justifiez pour chaque instance en donnant une génération pour le même mot par chaque grammaire, ou une preuve que l'intersection des langages est vide.
    $$\begin{array}{rcl}
      I_1 &=& \langle G_1, H_1 \rangle, \text{ avec :}\\
      
      G_1 &=&  \left\langle \Sigma, \{S\}, S, \left\{S \rightarrow aaSbb \mid \varepsilon\right\} \right\rangle \vspace{1mm}\\

      H_1 &=&  \left\langle \Sigma, \{T\}, T, \left\{T \rightarrow aaTbb \mid ab\right\} \right\rangle\\
    \end{array}
    \quad\quad
    \begin{array}{rcl}
      I_2 &=& \langle G_2, H_2 \rangle, \text{ avec :}\\
      
      G_2 &=&  \left\langle \Sigma, \{S\}, S, \left\{S \rightarrow aSb \mid ab\right\} \right\rangle \vspace{1mm}\\

      H_2 &=&  \left\langle \Sigma, \{T\}, T, \left\{T \rightarrow aT \mid bT \mid \varepsilon\right\} \right\rangle\\
    \end{array}$$
  \end{question}
  \begin{correction}
    \begin{itemize}
    \item $I_1$ est négative : $\mathcal{L}(G_1) = \{a^n b^n \mid n \text{ pair} \}$ et $\mathcal{L}(H_1) = \{a^n b^n \mid n \text{ impair} \}$, donc $\mathcal{L}(G_1) \cap \mathcal{L}(H_1) = \emptyset$
    \item $I_2$ est positive : on a $S \vdash ab$ et $T \vdash aT \vdash abT \vdash ab$, donc $ab \in \mathcal{L}(G_2) \cap \mathcal{L}(H_2)$.
    \end{itemize}
  \end{correction}

  On s'intéresse à la transformation $f$ qui à toute instance du PCP $D = \left\{ \VDomino{u_1}{v_1}, ..., \VDomino{u_n}{v_n} \right\}$
  associe l'instance $f\left(D\right) = \left\langle G, H \right\rangle$ du problème \textsc{Intersect-Alg} définie par :
  $$
  \begin{array}{rcl}
    G &=&  \left\langle \Sigma \cup D, \{S\}, S, \left\{\begin{array}{rrlllll}
    S &\rightarrow& u_1 \cdot S \cdot \VDomino[5mm]{u_1}{v_1} &\mid& ... &\mid& u_n \cdot S \cdot \VDomino[5mm]{u_n}{v_n}\\
    &\mid& u_1 \cdot \VDomino[5mm]{u_1}{v_1} &\mid& ... &\mid& u_n \cdot \VDomino[5mm]{u_n}{v_n}
    \end{array}\right\} \right\rangle \vspace{1mm}\\

    H &=&  \left\langle \Sigma \cup D, \{T\}, T, \left\{\begin{array}{rrlllll}
    T &\rightarrow& v_1 \cdot T \cdot \VDomino[5mm]{u_1}{v_1} &\mid& ... &\mid& v_n \cdot T \cdot \VDomino[5mm]{u_n}{v_n}\\
    &\mid& v_1 \cdot \VDomino[5mm]{u_1}{v_1} &\mid& ... &\mid& v_n \cdot \VDomino[5mm]{u_n}{v_n}
    \end{array}\right\} \right\rangle \vspace{1mm}\\
  \end{array}$$

  \emph{Attention : } Les dominos de $D$ sont considérés comme des symboles terminaux des grammaires $G$ et $H$ : chacun est considéré comme un symbole indivisible.

  
  \begin{question}
  \item Soit $D_3 = \left\{
    \VDomino[8mm]{ab}{a},
    \VDomino[8mm]{a}{b},
    \VDomino[8mm]{b}{ab}
    \right\}$. On note $\langle G_3, H_3 \rangle = f(D_3)$.
    
    Donnez la définition complète des grammaires $G_3$ et $H_3$.

  \end{question}
  \begin{correction}
    $f\left(D_3\right) = \langle G_3, H_3 \rangle$, avec :
    $$
    \begin{array}{rcl}
      G_3 &=&  \left\langle \Sigma \cup D_3, \{S\}, S, \left\{\begin{array}{rrlllll}
      S &\rightarrow&
      ab \cdot S \cdot \VDomino[8mm]{ab}{a} &\mid&
      a \cdot S \cdot \VDomino[8mm]{a}{b} &\mid&
      b \cdot S \cdot \VDomino[8mm]{b}{ab}\\&\mid&
      ab \cdot \VDomino[8mm]{ab}{a}&\mid&
      a \cdot \VDomino[8mm]{a}{b}  &\mid&
      b \cdot \VDomino[8mm]{b}{ab}
      \end{array}\right\} \right\rangle \vspace{1mm}\\
      
      H_3 &=&  \left\langle \Sigma \cup D_3, \{T\}, T, \left\{\begin{array}{rrlllll}
      T &\rightarrow&
      a \cdot T \cdot \VDomino[8mm]{ab}{a} &\mid&
      b \cdot T \cdot \VDomino[8mm]{a}{b} &\mid&
      ab \cdot T \cdot \VDomino[8mm]{b}{ab}\\&\mid&
      a \cdot \VDomino[8mm]{ab}{a}&\mid&
      b \cdot \VDomino[8mm]{a}{b}  &\mid&
      ab \cdot \VDomino[8mm]{b}{ab}
      \end{array}\right\} \right\rangle \vspace{1mm}\\
    \end{array}$$
  \end{correction}

  \begin{question}
  \item\label{exo:reductions/computability/intersection:d}
    Expliquez comment on peut en déduire une séquence de dominos qui constitue une solution à l'instance $D_3$, à partir de n'importe quel mot de $\mathcal{L}(G_3) \cap \mathcal{L}(H_3)$. 

    Illustrez la méthode sur le mot $w = abab \cdot 
    \VDomino[8mm]{b}{ab} \cdot
    \VDomino[8mm]{a}{b} \cdot
    \VDomino[8mm]{ab}{a}$
    que l'on affirme appartenir à $\mathcal{L}(G_3)$ et à $\mathcal{L}(H_3)$.
  \end{question}
  \begin{correction}
    Soit $w \in \mathcal{L}(G_3) \cap \mathcal{L}(H_3)$. D'après les grammaires $G_3$ et $H_3$, $w$ est de la forme
    
    $$w
    = u_{i_1} \cdots u_{i_k} \cdot \VDomino[8mm]{u_{i_k}}{v_{i_k}} \cdots \VDomino[8mm]{u_{i_1}}{v_{i_1}}
    = v_{j_1} \cdots v_{j_l} \cdot \VDomino[8mm]{u_{j_l}}{v_{j_l}} \cdots \VDomino[8mm]{u_{j_1}}{v_{j_1}}$$

    Par unicité de la décomposition du mot $w$ en symboles de $\Sigma \cup D$,
    on a $k=l$ et pour tout $i \in \{1, ..., k\}$, $i_{i} = j_{i}$.
    De plus, $u_{i_1} \cdots u_{i_k} = v_{j_1} \cdots v_{j_l}$.
    On en déduit la solution de $D$ :
    $$\VDomino[8mm]{u_{i_1}}{v_{i_1}} \cdots \VDomino[8mm]{u_{i_k}}{v_{i_k}}.$$
    
    À partir du mot $w = abab \cdot \VDomino[8mm]{b}{ab} \cdot \VDomino[8mm]{a}{b} \cdot \VDomino[8mm]{ab}{a}$ donné,
    on déduit la solution :
    $$
     \VDomino[8mm]{ab}{a} \cdot \VDomino[8mm]{a}{b} \cdot \VDomino[8mm]{b}{ab}.
    $$
  \end{correction}

  \begin{question}
  \item\label{exo:reductions/computability/intersection:e}
    Soit $D$ une instance du PCP et $\langle G, H \rangle = f(D)$.
    On suppose qu'il existe une solution $\VDomino{u_{i_1}}{v_{i_1}} \cdots \VDomino{u_{i_k}}{v_{i_k}}$ à $D$.

    Expliquez comment cette solution permet de construire un mot $w \in \mathcal{L}(G) \cap \mathcal{L}(H)$. 
    Justifiez en décrivant une génération de $w$ par chacune des deux grammaires.

  \end{question}
  \begin{correction}
    On a, pour $G$ :
    $$S \vdash u_{i_1} \cdot S \cdot \VDomino[8mm]{u_{i_1}}{v_{i_1}} \vdash ... \vdash
    u_{i_1} \cdots u_{i_{k-1}} \cdot S \cdot \VDomino[8mm]{u_{i_{k-1}}}{v_{i_{k-1}}} \cdots \VDomino[8mm]{u_{i_1}}{v_{i_1}}
    \vdash u_{i_1} \cdots u_{i_k} \cdot \VDomino[8mm]{u_{i_k}}{v_{i_k}} \cdots \VDomino[8mm]{u_{i_1}}{v_{i_1}}$$
    et pour $H$ :
    $$T \vdash v_{i_1} \cdot T \cdot \VDomino[8mm]{u_{i_1}}{v_{i_1}} \vdash ... \vdash
    v_{i_1} \cdots v_{i_{k-1}} \cdot T \cdot \VDomino[8mm]{u_{i_{k-1}}}{v_{i_{k-1}}} \cdots \VDomino[8mm]{u_{i_1}}{v_{i_1}}
    \vdash v_{i_1} \cdots v_{i_k} \cdot \VDomino[8mm]{u_{i_k}}{v_{i_k}} \cdots \VDomino[8mm]{u_{i_1}}{v_{i_1}}.$$
    
    Comme $\VDomino[8mm]{u_{i_1}}{v_{i_1}} \cdots \VDomino[8mm]{u_{i_k}}{v_{i_k}}$ est solution, 
    $u_{i_1} \cdots u_{i_{k}} = v_{i_1} \cdots v_{i_{k}}$, donc
    $$u_{i_1} \cdots u_{i_k} \cdot \VDomino[8mm]{u_{i_k}}{v_{i_k}} \cdots \VDomino[8mm]{u_{i_1}}{v_{i_1}}
    = v_{i_1} \cdots v_{i_k} \cdot \VDomino[8mm]{u_{i_k}}{v_{i_k}} \cdots \VDomino[8mm]{u_{i_1}}{v_{i_1}} \in \mathcal{L}(G) \cap \mathcal{L}(H).$$
  \end{correction}

  \begin{question}
  \item Concluez que le problème \textsc{Intersect-Alg} est indécidable. Précisez comment les questions précédentes permettent de déduire ce résultat.
  \end{question}
  \begin{correction}
    D'après les questions \ref{exo:reductions/computability/intersection:d} et \ref{exo:reductions/computability/intersection:e},
    $D$ est une instance positive de PCP si, et seulement si $f(D)$ est une instance positive de \textsc{Intersect-Alg}.
    De plus, la fonction $f$ est calculable. On en déduit que $f$ est une réduction par mappage de PCP vers \textsc{Intersect-Alg}, c'est-à-dire
    $\textsc{PCP} \le_m \textsc{Intersect-Alg}$. Cela prouve que \textsc{Intersect-Alg} hérite de l'indécidabilité du PCP.
  \end{correction}

\end{exercice}

\endgroup
\endinput
