% SPDX-License-Identifier: CC-BY-SA-4.0
% Author: Matthieu Perrin
% Part: Introduction
% Section: Words and languages
% Exercise: Words

\begingroup

\newcommand\ERR{\textsc{err}}

\begin{exercice}[Confluence d'un système de réécriture]

  On rappelle qu'un système de réécriture de mots $(\Sigma^\star, \to)$ est \emph{localement confluent}
  si, pour tout triplet de mots $u,v,w\in\Sigma^\star$,
  $$
  (u \to v \land u \to w) \;\Rightarrow\; \exists z,\; v \to^\star z \land w \to^\star z.
  $$
  On dit alors que la paire $\langle v, w \rangle$ est \emph{joignable}.
  Une paire $\langle v, w \rangle$ est dite \emph{critique}
  lorsqu'elle provient de deux règles dont 
  \begin{enumerate}
  \item les membres gauches se \emph{chevauchent} dans un même mot $u$,
  \item le membre gauche de l'une apparaît comme un préfixe de $u$, et
  \item celui de l'une ou de l'autre comme un suffixe.
  \end{enumerate}
  Pour déterminer si un système de réécriture est localement confluent,
  il suffit de vérifier que toutes ses paires critiques sont joignables.


  Le but de cet exercice est de construire, étape par étape,
  une réduction du problème de l'arrêt
  vers le problème suivant, et d'en déduire son indécidabilité.

  \Probleme{Confluence locale}{
    Un système de réécriture $\rightarrow$ de mots sur un alphabet $\Sigma$,
    défini par un ensemble fini de règles $\{l_1\hookrightarrow r_1,\dots,l_n\hookrightarrow r_n\}$.
  }{
    Le système de réécriture $\rightarrow$ est-il localement confluent ?
  }

  Considérons les deux systèmes de réécriture suivants :
  $$
  R_1 =
  \left\{\begin{array}{r@{~\hookrightarrow~}l}
  S & Pabaab\\
  S & I     \\
  P a & I   \\
  P b & P   \\
  I a & P   \\
  I b & I
  \end{array}\right.
  \quad\quad
  R_2 =
  \left\{\begin{array}{r@{~\hookrightarrow~}l}
  S & Pabaaab\\
  S & I     \\
  P a & I   \\
  P b & P   \\
  I a & P   \\
  I b & I
  \end{array}\right.
  $$
  \begin{question}
  \item Les systèmes $R_1$ et $R_2$ sont-ils localement confluent ?
  \end{question}
  \begin{correction}
    La seule paire critique de chacun des deux systèmes provient du symbole $S$, avec
    $S \to Pv$ et  $S \to I$.
    Le système est localement confluent ssi ces deux branches sont joignables. Comme $I$ est en forme normale,
    il faut $Pv\to^\star I$.
    
    \begin{itemize}
    \item Pour $R_1$, on a $Pabaab \rightarrow Ibaab \rightarrow Iaab \rightarrow Pab \rightarrow Ib \rightarrow I$, donc $R_1$ est localement confluent.
    \item Pour $R_2$, on a $Pabaaab \rightarrow Ibaaab \rightarrow Iaaab \rightarrow Paab \rightarrow Iab \rightarrow Pb \rightarrow P$,
      donc $R_2$ ne l'est pas.
    \end{itemize}
  \end{correction}

  On définit maintenant la fonction suivante :
  $$
  f : \left\{
  \begin{array}{rcl}
    \{a,b\}^\star &\to& \textsc{Confluence locale}_{\mathcal{I}}\\[0.2em]
    v &\mapsto&
    \left\{
    \begin{array}{r@{~\hookrightarrow~}l}
      S & P v \\
      S & I   \\
      P a & I \\
      P b & P \\
      I a & P \\
      I b & I
    \end{array}
    \right.
  \end{array}\right.
  $$

  \ifcorrection{\pagebreak}
  \begin{question}
  \item 
    Pour quel problème \textsc{Mystère} la fonction $f$ définit-elle une réduction vers
    \textsc{Confluence locale} ?  
    (Autrement dit, quel est le problème \textsc{Mystère} tel que
    $v\in \textsc{Mystère} \iff f(v)$ est localement confluent ?)
  \end{question}
  \begin{correction}
    En général, $f(v)$ est localement confluent ssi $Pv\to^\star I$. 
    Or la suite de mots de la réduction en forme normale de $Pv$ correspond à la suite de configurations dans l'exécution de
    l'automate suivant sur le mot $v$. Ainsi, $Pv$ et $I$ sont joignables ssi $v$ est reconnu par l'automate. 
    \begin{center}
      \begin{tikzpicture}[turingMachine]
        \state[initial above] (0) at (0,1) {$P$}; 
        \state[accepting]     (1) at (1,1) {$I$}; 

        \path (0) edge[bend left] node {$a$} (1);
        \path (1) edge[bend left] node {$a$} (0);
        \path (0) edge[loop left] node {$b$} (0);
        \path (1) edge[loop right] node {$b$} (1);
      \end{tikzpicture}
    \end{center}
    L'automate reconnaît le langage $\textsc{Mystère} = \{u \in \{a, b\}^\star \mid |u|_a \text{ est impair}\}$.
  \end{correction}
  
  \ifnotcorrection{\newpage}
  \begin{question}
  \item On cherche maintenant à reproduire la même idée avec une machine de Turing.
    Pour tout mot $v\in{a,b}^\star$, définir un système de réécriture $g(v)$ sur un alphabet
    $\Delta$ (à préciser) tel que la paire critique issue de $S$ est joignable si,
    et seulement si, le mot $v$ est \emph{engendré} par la machine de Turing suivante.

    \begin{center}
      \begin{tikzpicture}[turingMachine]
        \state[initial above, accepting] (0) at (0,1) {$0$}; 
        \state[]                         (1) at (1,1) {$1$}; 

        \path (0) edge[bend left] node {\smTMtransR{\blank}{a}} (1);
        \path (1) edge[bend left] node {\smTMtransL{\blank}{b}} (0);
        \path (0) edge[loop left] node {\smAlign{\smTMtransL{a}{a}\smTMtransL{b}{b}}} (0);
        \path (1) edge[loop right] node {\smAlign{\smTMtransR{a}{a}\smTMtransR{b}{b}}} (1);
      \end{tikzpicture}
    \end{center}
  \end{question}
  \begin{correction}
    On se place sur l'alphabet $\Delta = \{\llparenthesis, \rrparenthesis, S, a, b, \blank, 0, 1\}$. 
    $$
    g(v) = 
    \left\{\begin{array}{r@{~\hookrightarrow~}ll}
    S               & \llparenthesis0\rrparenthesis  & \text{configuration initiale}\\
    S               & \llparenthesis0v\rrparenthesis & \text{configuration finale}\\
    \llparenthesis         & \llparenthesis\blank    & \text{ajout d'un caractère blanc à gauche}\\
    \llparenthesis\blank   & \llparenthesis          & \text{suppression d'un caractère blanc à gauche}\\
    \rrparenthesis         & \blank \rrparenthesis   & \text{ajout d'un caractère blanc à droite}\\
    \blank \rrparenthesis  & \rrparenthesis          & \text{suppression d'un caractère blanc à droite}\\
    0\blank         & a1               & \text{transition $0\xrightarrow{\smTMtransR{\blank}{a}} 1$}\\
    1a              & a1               & \text{transition $1\xrightarrow{\smTMtransR{a}{a}} 1$}\\
    1b              & b1               & \text{transition $1\xrightarrow{\smTMtransR{b}{b}} 1$}\\
    \alpha 1 \blank & 0 \alpha b & \forall \alpha \in \{a, b, \blank\} : \text{transition $1\xrightarrow{\smTMtransL{\blank}{b}} 0$}\\ 
    \alpha 0 a      & 0 \alpha a & \forall \alpha \in \{a, b, \blank\} : \text{transition $0\xrightarrow{\smTMtransL{a}{a}} 0$}\\ 
    \alpha 0 b      & 0 \alpha b & \forall \alpha \in \{a, b, \blank\} : \text{transition $0\xrightarrow{\smTMtransL{b}{b}} 0$}
    \end{array}\right.
    $$
  \end{correction}
  
  \ifcorrection{\newpage}
  \begin{question}
  \item Listez toutes les paires critiques du système $g(v)$ selon les trois catégories suivantes
    et justifiez qu'il n'en existe pas d'autre :
    \begin{enumerate}
    \item\label{confluence:critical:S} la paire critique issue du symbole $S$ ;
    \item\label{confluence:critical:invalid} les paires critiques provenant de mots contenant plus d'un \emph{symbole d'état}
      (ces mots ne représentent pas une configuration valide de la machine, et ne peuvent pas être atteints depuis $S$) ;
    \item\label{confluence:critical:blank} les paires critiques provenant de la gestion des symboles blancs aux bords du ruban.
    \end{enumerate}
  \end{question}
  \begin{correction}
    \begin{enumerate}
    \item La seule paire critique issue de $S$ est $\langle \llparenthesis 0 \rrparenthesis, \llparenthesis 0 v \rrparenthesis  \rangle$.
    \item On cherche des mots $\alpha q_1 \beta q_2 \gamma$ (resp. $q_1 \beta q_2 \gamma$), tels que $\alpha q_1 \beta$ (resp. $\alpha q_1 \beta$)
      et $\beta q_2 \gamma$ sont les membres gauches de règles venant des transitions. Par exemple, $1a0a$ donne la paire $\langle a10a, 10aa  \rangle$,
      et $\blank 0 a 0 b$ donne la paire $\langle 0 \blank a 0 b, \blank 0 0 a b \rangle$.
    \item On a les paires issues de l'intéraction entre $\llparenthesis\blank$ et les règles commençant par un $\blank$
      (par exemple, $\llparenthesis\blank 0 a$ donne la paire critique $\langle \llparenthesis 0 a, \llparenthesis 0 \blank a\rangle$),
      et les paires issues de l'intéraction entre $\blank \rrparenthesis$ et les règles finissant par un $\blank$
      (par exemple, $0 \blank \rrparenthesis$ donne la paire critique $\langle 0 \rrparenthesis, a1 \rrparenthesis\rangle$).
    \end{enumerate}
  \end{correction}
  
  \begin{question}
  \item Montrez que les paires critiques de la catégorie (\ref{confluence:critical:blank}) sont joignables.
  \end{question}
  \begin{correction}
    Le système de réécriture est joignable ssi le mot est engendré par la machine de Turing ci-dessus, donc
    $\{a^nb^n \mid n\in \mathbb{N}\} \le_m \textsc{Confluence locale}$. 
  \end{correction}

  \begin{correction}
    \begin{enumerate}
    \item La seule paire critique issue de $S$ est $\langle \llparenthesis 0 \rrparenthesis, \llparenthesis 0 v \rrparenthesis  \rangle$.
    \item On cherche des mots $\alpha q_1 \beta q_2 \gamma$ (resp. $q_1 \beta q_2 \gamma$), tels que $\alpha q_1 \beta$ (resp. $\alpha q_1 \beta$)
      et $\beta q_2 \gamma$ sont les membres gauches de règles venant des transitions. Par exemple, $1a0a$ donne la paire $\langle a10a, 10aa  \rangle$,
      et $\blank 0 a 0 b$ donne la paire $\langle 0 \blank a 0 b, \blank 0 0 a b \rangle$.
    \item On a les paires issues de l'intéraction entre $\llparenthesis\blank$ et les règles commençant par un $\blank$
      (par exemple, $\llparenthesis\blank 0 a$ donne la paire critique $\langle \llparenthesis 0 a, \llparenthesis 0 \blank a\rangle$),
      et les paires issues de l'intéraction entre $\blank \rrparenthesis$ et les règles finissant par un $\blank$
      (par exemple, $0 \blank \rrparenthesis$ donne la paire critique $\langle 0 \rrparenthesis, a1 \rrparenthesis\rangle$).
    \end{enumerate}
  \end{correction}

  Pour traiter les paires critiques de la catégorie~(\ref{confluence:critical:invalid}),
  on introduit deux symboles supplémentaires (le marqueur `?' et `\ERR{}') et les règles suivantes : 
  $$
  \left\{\begin{array}{r@{~\hookrightarrow~}ll}
  q   & q ?  & \text{pour tout état } q \in \{0,1\},\\
  ? q & \ERR & \text{pour tout état } q \in \{0,1\},\\
  ? \alpha & \alpha ? & \text{pour tout } \alpha\in\Delta,\\
  ? & \varepsilon,\\
  \ERR \alpha & \ERR & \text{pour tout } \alpha\in\Delta,\\
  \alpha \ERR & \ERR & \text{pour tout } \alpha\in\Delta.\\
  \end{array}\right.
  $$

  On admettra que toutes les paires critiques introduites par ces nouvelles règles sont joignables.

  \begin{question}
  \item Montrez que tout mot $u$ contenant au moins deux symboles d'état peut se réécrire en une ou plusieurs étapes en $\ERR$,
    et en déduire que toutes les paires critiques de la catégorie~(\ref{confluence:critical:invalid}) sont joignables.
  \end{question}
  \begin{correction}
    Les paires critiques de la troisième catégorie utilisent toutes la règle $\llparenthesis \blank \hookrightarrow \llparenthesis$
    ou la règle $\blank\rrparenthesis \hookrightarrow \rrparenthesis$. Or ces deux règles sont réversibles. 
  \end{correction}

  \ifcorrection{\pagebreak}
  \begin{question}
  \item Entre quels problèmes la réduction construite dans les questions précédentes établit-elle un lien, et dans quel sens ?  
  \end{question}
  \begin{correction}
    Soit $u = v q_1 \alpha_1 ... \alpha_n q_2 w$ un mot contenant au moins deux symboles d'état. On a :
    $$
    \begin{array}{rcl}
      u &\rightarrow& v q_1 ? \alpha_1 ... \alpha_n q_2 w \\
      &\rightarrow& v q_1 \alpha_1 ? ... \alpha_n q_2 w \\
      &\rightarrow^\star& v q_1 \alpha_1 ... \alpha_n ? q_2 w \\
      &\rightarrow& v q_1 \alpha_1 ... \alpha_n \ERR w \\
      &\rightarrow^\star& v q_1 \alpha_1 ... \alpha_n \ERR \\
      &\rightarrow^\star& v q_1 \ERR \\
      &\rightarrow& v \ERR \\
      &\rightarrow^\star& \ERR
    \end{array}
    $$

    Toutes les paires critiques de la catégorie (2) sont formées de mots contenant deux symboles d'état, donc elles sont toujours joignables en \ERR. 
  \end{correction}

  \begin{question}
  \item En généralisant cette construction à une machine de Turing arbitraire $M$ \emph{configurée pour la reconnaissance}
    (qui, si elle s'arrête, efface entièrement son ruban et entre dans un état final $q_f$),
    montrez que le problème \textsc{Confluence locale} est indécidable
    par réduction du problème de l'arrêt.
  \end{question}
  \begin{correction}
    On pose la réduction suivante : 
    $$
    r : \left\{\begin{array}{rcl}
    \textsc{halt}_\mathcal{I} & \rightarrow & \textsc{Confluence locale}_\mathcal{I}\\
    \langle \langle \Sigma, \Gamma, \blank, Q, q_0, \{q_f\}, \rightarrow \rangle , v \rangle & \mapsto &
    \left\{\begin{array}{r@{~\hookrightarrow~}ll}
    S                     & \llparenthesis q_0 v \rrparenthesis & \text{configuration initiale}\\
    S                     & \llparenthesis q_f \rrparenthesis   & \text{configuration finale}\\
    \llparenthesis        & \llparenthesis\blank                & \text{ajout de \blank\ à gauche}\\
    \llparenthesis\blank  & \llparenthesis                      & \text{suppression de \blank\ à gauche}\\
    \rrparenthesis        & \blank \rrparenthesis               & \text{ajout de \blank\ à droite}\\
    \blank \rrparenthesis & \rrparenthesis                      & \text{suppression de \blank\ à droite}\\
    q \alpha              & \beta q'                            & \forall q\xrightarrow{\smTMtransR{\alpha}{\beta}} q'\\
    \gamma q \alpha       & q' \gamma \beta                     & \forall q\xrightarrow{\smTMtransL{\alpha}{\beta}} q', \gamma\in \Gamma\\
    q   & q ?  & \forall q \in Q\\
    ? q & \ERR & \forall q \in Q\\
    ? \gamma & \gamma ? & \forall \gamma \in \Gamma\\
    ? & \varepsilon\\
    \ERR\ \alpha & \ERR & \forall \alpha\in \Gamma \cup Q \cup \{\llparenthesis, \rrparenthesis, ?, \ERR\}\\
    \alpha\ \ERR & \ERR & \forall \alpha\in \Gamma \cup Q \cup \{\llparenthesis, \rrparenthesis, ?, \ERR\}
    \end{array}\right.
    \end{array}\right.
    $$

    La paire critique issue de $S$ est joignable ssi $q_0 v \rightarrow^\star q_f$, ssi $M$ s'arrête sur $v$.
    Les autres paires critiques sont joignables pour la même raison que dans l'exemple précédent.
    Ainsi, $r\langle M, v \rangle$ est localement confluente si et seulement si $M$ s'arrête sur $v$. 
    La transformation $r$ est calculable, donc $$\textsc{Halt}\ \le_m\ \textsc{Confluence locale}.$$
    Finalement, \textsc{Confluence locale} est indécidable.
  \end{correction}

\end{exercice}

\endgroup
\endinput
