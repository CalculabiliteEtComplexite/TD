% SPDX-License-Identifier: CC-BY-SA-4.0
% Author: Matthieu Perrin
% Part: Introduction
% Section: Words and languages
% Exercise: Words

\begingroup

\begin{exercice}[Le problème SAT et la taille de ses clauses]

  Voici les définitions des problèmes \textsc{Sat} et \textsc{3-Sat}.

  \Probleme{Sat}{
    Une formule booléenne $\varphi$ en forme normale conjonctive (FNC), construite à partir d’un ensemble $X = \{x_1, x_2, \dots, x_n\}$ de $n$ variables.
  }{
    Existe-t-il une affectation (Vrai/Faux) de chacune des variables de $X$ qui rend $\varphi$ satisfiable ?
  }

  On sait que \textsc{Sat} est NP-complet (théorème de Cook–Levin, 1971).  
  La question qui se pose est de savoir si le problème devient plus simple si l’on se limite aux instances dont les clauses ont une taille « limitée ».  
  On commence avec la taille~3.

  \Probleme{3-Sat}{
    Une formule booléenne $\varphi$ en forme normale conjonctive (FNC), construite à partir d’un ensemble $X = \{x_1, x_2, \dots, x_n\}$ de $n$ variables,
    et dans laquelle chaque clause contient au plus trois littéraux.
  }{
    Existe-t-il une affectation (Vrai/Faux) de chacune des variables de $X$ qui rend $\varphi$ satisfiable ?
  }

  \begin{question}
    \item Montrer que \textsc{3-Sat} est dans NP.

    \medskip
    On veut maintenant montrer que \textsc{Sat} se réduit en temps polynomial à \textsc{3-Sat}, ce que l’on note
    \[
    \textsc{Sat} \le_P \textsc{3-Sat}.
    \]
    Pour cela, partant d’une instance quelconque $\varphi$ de \textsc{Sat}, on construit une instance $\varphi'$ de \textsc{3-Sat} de la manière suivante :
    \begin{enumerate}
    \item Pour toute clause $C$ de $\varphi$ qui contient au plus trois littéraux, on fixe $C' = C$ dans $\varphi'$.
    \item Pour toute clause $C = (a_1 \lor a_2 \lor \dots \lor a_k)$ de $\varphi$ contenant $k \ge 4$ littéraux,
      on construit dans $\varphi'$ l’expression :
      \[
      E'_C = (a_1 \lor a_2 \lor y_1)
      \land (\neg y_1 \lor a_3 \lor y_2)
      \land (\neg y_2 \lor a_4 \lor y_3)
      \land \dots
      \land (\neg y_{k-4} \lor a_{k-2} \lor y_{k-3})
      \land (\neg y_{k-3} \lor a_{k-1} \lor a_k),
      \]
      où les $y_i$ sont de nouvelles variables.
    \item La formule $\varphi'$ relie toutes les clauses et expressions listées ci-dessus par des conjonctions logiques.
    \end{enumerate}

    \item Illustrer cette réduction dans le cas où
    \[
    \varphi = (x_1 \lor \neg x_2)
    \land (x_1 \lor x_2 \lor \neg x_3 \lor \neg x_4 \lor x_5 \lor \neg x_6 \lor x_7)
    \land (\neg x_2 \lor x_3 \lor \neg x_5).
    \]

    \item Dans le cas général, cette réduction se fait-elle bien en temps polynomial ?

    \item Montrer que si une clause $C$ à $k \le 3$ littéraux est satisfaite dans $\varphi$, alors sa clause correspondante $C'$ est satisfaite dans $\varphi'$.

    \medskip
    Soit $C = (a_1 \lor a_2 \lor \dots \lor a_k)$ une clause à $k \ge 4$ littéraux dans $\varphi$ et $E'_C$ son expression correspondante dans $\varphi'$.
    Supposons que $C$ soit satisfaite. Il existe donc au moins une variable, que l’on appellera $a_i$, qui rend $C$ vraie.

    \item Montrer que l’affectation
    \[
    y_1 = y_2 = \dots = y_{i-2} = \text{Vrai}
    \quad \text{et} \quad
    y_{i-1} = y_i = \dots = y_{k-3} = \text{Faux}
    \]
    satisfait $E'_C$.

    \item Que déduit-on des questions précédentes~?

    \medskip
    Une clause $C'$ de $\varphi'$ construite à partir d’une clause $C$ à $k \le 3$ littéraux de $\varphi$ est appelée une \emph{petite clause}.

    \item Montrer que si une petite clause $C'$ est satisfaite dans $\varphi'$, alors la clause $C$ dont elle provient est satisfaite dans $\varphi$.

    \item Montrer que si une expression $E'_C$ est satisfaite dans $\varphi'$, alors au moins un des $a_i$ doit être vrai.  
    En déduire que si une expression $E'_C$ est satisfaite dans $\varphi'$, alors la clause $C$ dont elle provient est satisfaite dans $\varphi$.

    \item Que déduit-on des questions précédentes~?

    \item Que déduit-on des questions~3, 6 et~9~?

    \item Que déduit-on des questions~1 et~10~?

    \item Montrer que le problème \textsc{3-Sat} limité aux clauses de taille exactement égale à~3 est NP-complet.  
    \emph{Indice :} adapter la réduction précédente pour traiter les clauses de taille $< 3$.

    \item Montrer que, pour tout $k \ge 3$, le problème \textsc{k-Sat} limité aux clauses de taille au plus égale à $k$ est NP-complet.  
    Montrer qu’il en est de même pour le problème \textsc{k-Sat} limité aux clauses de taille exactement égale à $k$.

    \medskip
    Il reste à résoudre la question de la complexité du problème \textsc{2-Sat} (c’est-à-dire \textsc{Sat} limité aux clauses de taille~2).

    \item Une clause de taille~2, $C = (a \lor b)$, peut aussi s’écrire comme la conjonction de deux implications.  
    Lesquelles~?

    \medskip
    Partant de ce constat, on peut montrer que \textsc{2-Sat} est dans~P, en passant par le \emph{graphe des implications} $G = (V, A)$, où, pour une instance $\varphi$ de \textsc{2-Sat} :
    \begin{itemize}
    \item $V$ est l’ensemble des littéraux de $\varphi$ ;
    \item $A$ est l’ensemble des arcs $(\ell, \ell')$ correspondant à une implication de la forme $\ell \Rightarrow \ell'$.
    \end{itemize}

    \item Dessiner le graphe $G_\varphi$ des implications de la formule :
    \[
    \varphi = (\neg x_1 \lor x_2) \land (\neg x_1 \lor \neg x_2) \land (x_1 \lor \neg x_3) \land (x_1 \lor x_3).
    \]

    \item En observant $G_\varphi$, montrer que $\varphi$ ne peut pas être satisfaite.

    \medskip
    \emph{Remarque :} on peut en réalité montrer qu’une formule $\varphi$ peut être satisfaite si et seulement si son graphe des implications $G_\varphi$ ne contient aucun circuit passant à la fois par $\ell$ et $\neg\ell$.  
    Cette propriété peut être vérifiée par un algorithme de graphes qui calcule, en temps polynomial, les composantes fortement connexes de $G_\varphi$.  
    Ainsi, \textsc{2-Sat} est dans~P : la construction puis l’analyse de $G_\varphi$ se font toutes deux en temps polynomial.
  \end{question}

\end{exercice}

\begin{correction}
  \begin{question}

  \item Le but ici est de donner des arguments « en français », sans trop formaliser.  
    L’appartenance à NP se justifie en deux étapes :
    \begin{enumerate}
      \item proposer une solution potentielle doit se faire en temps polynomial ;
      \item vérifier si cette solution convient doit aussi se faire en temps polynomial.
    \end{enumerate}
    Ici, comme pour \textsc{Sat} :
    \begin{itemize}
      \item proposer une solution $S$ consiste à donner une valeur (Vrai/Faux) à chaque $x_i$, ce qui se fait en $O(n)$, où $n$ est le nombre de variables ;
      \item vérifier si $S$ convient consiste à injecter ces valeurs dans la formule $\varphi$ et à en évaluer la vérité, ce qui se fait en $O(\ell)$, où $\ell$ est la taille (nombre de littéraux) de $\varphi$.
    \end{itemize}

  \item En rouge, ce qui a été modifié entre $\varphi$ et $\varphi'$.  
    \[
      \varphi' =
      (x_1 \lor \neg x_2)
      \land (x_1 \lor x_2 \lor y_1)
      \land (\neg y_1 \lor \neg x_3 \lor y_2)
      \land (\neg y_2 \lor \neg x_4 \lor y_3)
      \land (\neg y_3 \lor x_5 \lor y_4)
      \land (\neg y_4 \lor \neg x_6 \lor x_7)
      \land (\neg x_2 \lor x_3 \lor \neg x_5).
    \]

  \item Chaque clause « longue » (de taille $k>3$) est transformée en $k-2$ clauses de taille~3, en ajoutant $k-3$ nouvelles variables.  
    Si $\ell$ est la taille de $\varphi$, on a au plus $O(\ell)$ clauses longues, donc la taille de $\varphi'$ est au pire en $O(\ell^2)$.  
    Le nombre de variables dans $\varphi'$ est en $O(n+\ell)$, et le passage de $\varphi$ à $\varphi'$ se fait donc en temps polynomial.

  \item La clause étant inchangée, si elle est satisfaite dans $\varphi$, elle l’est nécessairement dans $\varphi'$.

  \item On peut illustrer sur l’exemple de la Question 2, en supposant que la variable $a_i$ correspond à $x_4$ (donc $x_4 = \text{Faux}$).  
    Si $a_i$ apparaît dans la première (resp. dernière) clause, alors mettre tous les $y_j$ à Faux (resp. à Vrai) suffit à satisfaire $E'_C$.  
    Dans les autres cas : la clause contenant $a_i$ est satisfaite, donc ni $\neg y_{i-2}$ ni $y_{i-1}$ (qui entourent $a_i$) ne sont contraints.  
    Les clauses « à gauche » de $a_i$ contiennent toutes un littéral $y_j$ positif ; on peut donc fixer $y_1, \dots, y_{i-2}$ à Vrai.  
    Symétriquement, les clauses « à droite » contiennent des littéraux $\neg y_j$ ; on peut donc fixer $y_{i-1}, \dots, y_{k-3}$ à Faux.  
    Cette affectation satisfait bien $E'_C$.

  \item Les réponses aux Questions 4 et 5 montrent que si $\varphi$ est satisfiable, alors $\varphi'$ l’est aussi.

  \item La clause étant inchangée, si elle est satisfaite dans $\varphi'$, elle l’est nécessairement dans $\varphi$.

  \item Raisonnons par contradiction.  
    Supposons que toutes les clauses d’une expression $E'_C$ soient satisfaites uniquement grâce aux $y_j$.  
    Alors, à cause de la première clause, $y_1$ doit être Vrai. En propageant, on déduit que tous les $y_j$ ($1 \le j \le k-3$) sont à Vrai.  
    Mais la dernière clause $(\neg y_{k-3} \lor a_{k-1} \lor a_k)$ serait alors insatisfaite, puisque $\neg y_{k-3}$ serait Faux.  
    On en déduit que si $E'_C$ est satisfaite, alors au moins un des littéraux $a_i$ doit être Vrai.

  \item Les réponses aux Questions 7 et 8 montrent que si $\varphi'$ est satisfiable, alors $\varphi$ l’est aussi.

  \item Les Questions 3, 6 et 9 montrent que l’on a bien une réduction polynomiale
    \[
      \textsc{Sat} \le_P \textsc{3-Sat},
    \]
    c’est-à-dire qu’à toute instance $\varphi$ de \textsc{Sat} on peut associer en temps polynomial une instance $\varphi'$ de \textsc{3-Sat}
    telle que $\varphi$ est satisfiable si et seulement si $\varphi'$ l’est.

  \item La Question 1 indique que \textsc{3-Sat} est dans NP, et la Question 10 que \textsc{Sat} se réduit à \textsc{3-Sat}.  
    Comme \textsc{Sat} est NP-complet, on en déduit que \textsc{3-Sat} est lui aussi NP-complet.

  \item Pour limiter les clauses à exactement trois littéraux, traitons les cas particuliers :
    \begin{itemize}
      \item Si une clause $C$ a taille 1, elle fixe la valeur d’une variable $x_i$ ; on peut alors simplifier la formule en conséquence.
      \item Si une clause $C$ a taille 2, par exemple $C = (a \lor b)$, on remplace dans $\varphi'$ par $C' = (a \lor b \lor b)$.  
            Clairement, $C$ est satisfaite $\Leftrightarrow$ $C'$ l’est.
    \end{itemize}
    Ainsi, on adapte la réduction précédente pour obtenir uniquement des clauses de taille exactement 3, sans changer la validité de la preuve.

  \item Pour tout $k \ge 3$, on peut réduire \textsc{k-Sat} vers \textsc{(k+1)-Sat} en transformant chaque clause
    \[
      C = (a_1 \lor a_2 \lor \dots \lor a_k)
      \quad\text{en}\quad
      C' = (a_1 \lor a_2 \lor \dots \lor a_k \lor a_k).
    \]
    Cette réduction se fait en temps polynomial, et $C$ est satisfaite $\Leftrightarrow$ $C'$ l’est.  
    Comme \textsc{k-Sat} est NP-complet, on déduit par récurrence que \textsc{k-Sat} est NP-complet pour tout $k \ge 3$.

  \item Une clause de taille 2, $C = (a \lor b)$, est équivalente à :
    \[
      (\neg a \Rightarrow b) \;\land\; (\neg b \Rightarrow a).
    \]

  \item Pour la formule
    \[
      \varphi = (\neg x_1 \lor x_2)
      \land (\neg x_1 \lor \neg x_2)
      \land (x_1 \lor \neg x_3)
      \land (x_1 \lor x_3),
    \]
    on obtient les implications suivantes :
    \begin{align*}
      (\neg x_1 \lor x_2) &\Rightarrow (x_1 \Rightarrow x_2) \land (\neg x_2 \Rightarrow \neg x_1),\\
      (\neg x_1 \lor \neg x_2) &\Rightarrow (x_1 \Rightarrow \neg x_2) \land (x_2 \Rightarrow \neg x_1),\\
      (x_1 \lor \neg x_3) &\Rightarrow (\neg x_1 \Rightarrow \neg x_3) \land (x_3 \Rightarrow x_1),\\
      (x_1 \lor x_3) &\Rightarrow (\neg x_1 \Rightarrow x_3) \land (\neg x_3 \Rightarrow x_1).
    \end{align*}
    Le graphe $G_\varphi$ des implications correspond à ces arcs (à une inversion près pour celui du bas !).

    \begin{center}
        \begin{tikzpicture}[x=20mm, y=10mm]
          \draw (1,1) node[graph node, text width=8mm] (t1) {$x_1$};
          \draw (2,1) node[graph node, text width=8mm] (t3) {$x_3$};
          \draw (3,1) node[graph node, text width=8mm] (f2) {$\lnot x_2$};
          \draw (1,0) node[graph node, text width=8mm] (t2) {$x_2$};
          \draw (2,0) node[graph node, text width=8mm] (f3) {$\lnot x_3$};
          \draw (3,0) node[graph node, text width=8mm] (f1) {$\lnot x_1$};

          \draw[thick, -latex] (f1) to (t3);
          \draw[thick, -latex] (f1) to (f3);
          \draw[thick, -latex] (t3) to (t1);
          \draw[thick, -latex] (f3) to (t1);
          \draw[thick, -latex] (t1) to (t2);
          \draw[thick, -latex] (t1) to[bend left] (f2);
          \draw[thick, -latex] (t2) to[bend right] (f1);
          \draw[thick, -latex] (f2) to (f1);

        \end{tikzpicture}
    \end{center}

    
  \item En observant $G_\varphi$, on constate :
    \begin{itemize}
      \item si $x_1 = \text{Vrai}$, alors $x_2 = \text{Faux}$, donc $x_1 = \text{Faux}$ ;
      \item si $x_1 = \text{Faux}$, alors $x_3 = \text{Vrai}$, donc $x_1 = \text{Vrai}$.
    \end{itemize}
    Ainsi, $x_1$ ne peut être ni Vrai ni Faux : la formule $\varphi$ est donc insatisfiable.
  
  \end{question}
\end{correction}

\endgroup
\endinput
