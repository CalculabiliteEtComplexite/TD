% SPDX-License-Identifier: CC-BY-SA-4.0
% Author: Matthieu Perrin
% Part: Introduction
% Section: Words and languages
% Exercise: Words

\begingroup

\begin{exercice}[Le problème \textsc{Cycle Hamiltonien}\footnote{\textit{Source :} \url{https://www.youtube.com/watch?v=kl2k_wgYYic}.}]

  On s'intéresse au problème suivant :


  \Probleme{Circuit Hamiltonien}{
    un graphe non orienté $G$.
  }{
    $G$ contient-il un cycle hamiltonien (c'est-à-dire un cycle qui passe une et une seule fois par chaque sommet de $G$)~?
  }

  Le problème \textsc{Cycle Hamiltonien} est donc l'équivalent non orienté du problème \textsc{Circuit Hamiltonien}.

  \begin{question}
  \item Montrer que \textsc{Cycle Hamiltonien} est dans NP.
  \end{question}
  \begin{correction}
    Un certificat naturel est un cycle simple listant chaque sommet exactement une fois (avec retour au point de départ).

    Pour vérifier le certificat, on contrôle que la liste contient $n$ sommets
    tous distincts ($O(n)$ avec marquage), que chaque paire consécutive est une
    arête du graphe ($O(n)$ via matrice d'adjacence ou table de hachage),
    et que le premier et le dernier sont adjacents ($O(1)$).
    La vérification est donc polynomiale.
    
    Ainsi, $\textsc{Cycle Hamiltonien} \in \text{NP}$.
  \end{correction}
  
  On veut maintenant démontrer qu'on peut réduire en temps polynomial le problème \textsc{Circuit Hamiltonien} vers \textsc{Cycle Hamiltonien}, c'est-à-dire :
  $$\textsc{Circuit Hamiltonien} \le_P \textsc{Cycle Hamiltonien}.$$
  
  Voici (sur un exemple) la réduction que l'on propose.
  À gauche, le graphe orienté $G$, à droite le graphe non orienté $G'$.

  \begin{center}
    \begin{tikzpicture}
      \draw (  90:2cm) node[graph node, circle] (v1) {$v_1$};
      \draw ( -30:2cm) node[graph node, circle] (v2) {$v_2$};
      \draw (-150:2cm) node[graph node, circle] (v3) {$v_3$};

      \draw[thick, -latex] (v1) -- (v2);
      \draw[thick, -latex] (v2) -- (v3);
      \draw[thick, -latex] (v3) -- (v1);
    \end{tikzpicture}
    \quad
    \quad
    \quad
    \begin{tikzpicture}
      \draw (  90:2cm)             node[graph node, circle] (v1)  {$v_1$};
      \draw (  90:2cm) +(-120:1cm) node[graph node, circle] (v1i) {$v_1^i$};
      \draw (  90:2cm) +( -60:1cm) node[graph node, circle] (v1o) {$v_1^o$};

      \draw ( -30:2cm)             node[graph node, circle] (v2)  {$v_2$};
      \draw ( -30:2cm) +( 120:1cm) node[graph node, circle] (v2i) {$v_2^i$};
      \draw ( -30:2cm) +( 180:1cm) node[graph node, circle] (v2o) {$v_2^o$};

      \draw (-150:2cm)             node[graph node, circle] (v3)  {$v_3$};
      \draw (-150:2cm) +(   0:1cm) node[graph node, circle] (v3i) {$v_3^i$};
      \draw (-150:2cm) +(  60:1cm) node[graph node, circle] (v3o) {$v_3^o$};

      \draw[thick] (v1i) -- (v1) -- (v1o);
      \draw[thick] (v2i) -- (v2) -- (v2o);
      \draw[thick] (v3i) -- (v3) -- (v3o);
      \draw[thick] (v1o) -- (v2i);
      \draw[thick] (v2o) -- (v3i);
      \draw[thick] (v3o) -- (v1i);
    \end{tikzpicture}
  \end{center}

  \begin{question}
  \item On souhaite montrer (d'abord sur l'exemple) que l'on répond \textsc{OUI} à \textsc{Circuit Hamiltonien} pour $G$
    si et seulement si on répond \textsc{OUI} à \textsc{Cycle Hamiltonien} pour $G'$.
  \end{question}
  \begin{correction}
    On part d'un graphe orienté $G=(V,A)$ et on construit un graphe non orienté $G'$ tel que :
    $$
    G \text{ a un circuit hamiltonien } \iff G' \text{ a un cycle hamiltonien}.
    $$
    Intuition : remplacer chaque sommet orienté par un \emph{gadget de direction} qui impose de le traverser dans un
    sens « entrée $\to$ sortie », et remplacer chaque arc $(u,v)$ par une arête reliant la sortie du gadget de $u$ à l'entrée du gadget de $v$.
  \end{correction}
  
  \begin{question}
  \item Clairement, un circuit hamiltonien existe dans le graphe $G$ (à gauche).  
    En déduire un cycle hamiltonien dans le graphe $G'$ (à droite).
  \end{question}
  \begin{correction}
    Si un circuit hamiltonien de $G$ visite $v_1 \to v_2 \to \dots \to v_n \to v_1$, alors dans $G'$ on parcourt successivement,
    pour chaque $v_i$, le gadget $v_i$ \emph{de l'entrée à la sortie}, puis on suit l'arête correspondant à l'arc $(v_i,v_{i+1})$
    vers l'entrée du gadget $v_{i+1}$. On obtient un cycle hamiltonien de $G'$.
  \end{correction}

  \begin{question}
  \item Trouver un cycle hamiltonien dans $G'$, et en déduire un circuit hamiltonien dans $G$.
  \end{question}
  \begin{correction}
    Un cycle hamiltonien de $G'$ doit traverser chaque gadget de $G'$ en suivant
    l'ordre imposé « entrée $\to$ milieu $\to$ sortie » (voir le gadget ci-dessous).
    Les seules arêtes quittant la sortie de $u$ mènent à l'entrée de gadgets $v$ tels que $(u,v)\in A$.
    En lisant ces transitions, on récupère un circuit hamiltonien orienté dans $G$.
  \end{correction}

  \begin{question}
  \item On cherche maintenant à rendre générale la réduction : indiquer, dans le cas général, comment construire le graphe non orienté $G'$ à partir d'un graphe orienté $G$.  
    Illustrer cette réduction sur le graphe orienté $G$ à trois sommets $v_1, v_2, v_3$ possédant les arcs suivants :
    $$(v_2, v_1),\quad (v_3, v_1),\quad (v_2, v_3).$$
  \end{question}
  \begin{correction}
    Pour chaque sommet $v\in V$, créer trois sommets $v^{\mathsf{in}}, v^{\mathsf{mid}}, v^{\mathsf{out}}$ et les relier en triangle :
    \[
    \{v^{\mathsf{in}}v^{\mathsf{mid}},\ v^{\mathsf{mid}}v^{\mathsf{out}},\ v^{\mathsf{in}}v^{\mathsf{out}}\}.
    \]
    (Graphe non orienté.)  
    Pour chaque arc $(u,v)\in A$, ajouter l'arête non orientée $u^{\mathsf{out}}v^{\mathsf{in}}$.
    \begin{itemize}
    \item \emph{Rôle du triangle :} forcer qu'un cycle hamiltonien qui visite les trois sommets du gadget (obligatoire) les parcoure consécutivement, et permette une « entrée » par $v^{\mathsf{in}}$ et une « sortie » par $v^{\mathsf{out}}$ (direction simulée).
    \end{itemize}

    \emph{Illustration demandée :} pour $V=\{v_1,v_2,v_3\}$ et $A=\{(v_2,v_1),(v_3,v_1),(v_2,v_3)\}$, créer les triangles
    $v_i^{\mathsf{in}}-v_i^{\mathsf{mid}}-v_i^{\mathsf{out}}-v_i^{\mathsf{in}}$ pour $i=1,2,3$, puis ajouter les arêtes
    $v_2^{\mathsf{out}}v_1^{\mathsf{in}}$, $v_3^{\mathsf{out}}v_1^{\mathsf{in}}$, $v_2^{\mathsf{out}}v_3^{\mathsf{in}}$.
  \end{correction}

  \begin{question}
  \item Pour un graphe orienté $G$ à $n$ sommets et $m$ arcs, quelle est la taille (nombre de sommets, nombre d'arêtes) du graphe $G'$~?  
    En déduire que la réduction se fait en temps polynomial.
  \end{question}
  \begin{correction}
    Si $G$ a $n$ sommets et $m$ arcs :
    \[
    |V(G')| = 3n,\qquad |E(G')| = 3n + m
    \]
    (trois arêtes internes par gadget, plus une arête par arc de $G$).  
    La construction se fait en temps $O(n+m)$ (polynomial).
  \end{correction}

  \begin{question}
  \item Montrer que si $G$ admet un circuit hamiltonien, alors $G'$ admet un cycle hamiltonien.
  \end{question}
  \begin{correction}
    Soit $v_1\to v_2\to\cdots\to v_n\to v_1$ un circuit hamiltonien dans $G$. Dans $G'$, pour chaque $v_i$ on parcourt
    $v_i^{\mathsf{in}} - v_i^{\mathsf{mid}} - v_i^{\mathsf{out}}$, puis l'arête $v_i^{\mathsf{out}}v_{i+1}^{\mathsf{in}}$ correspondant à l'arc $(v_i,v_{i+1})$.  
    On visite ainsi exactement une fois tous les sommets de $G'$ et on ferme le cycle : c'est un cycle hamiltonien.
  \end{correction}

  \begin{question}
  \item Montrer que si $G'$ admet un cycle hamiltonien, alors $G$ admet un circuit hamiltonien.
  \end{question}
  \begin{correction}
    Dans $G'$, tout cycle hamiltonien doit visiter les trois sommets de chaque gadget. Le triangle impose qu'ils soient pris consécutivement, et qu'on « entre » par $v^{\mathsf{in}}$ et « sorte » par $v^{\mathsf{out}}$ (sinon on ne peut pas connecter les gadgets sans répéter des sommets).  
    Les arêtes reliant gadgets sont exactement les $u^{\mathsf{out}}v^{\mathsf{in}}$ pour $(u,v)\in A$, donc l'ordre des gadgets lu au niveau des sorties/entrées code un circuit hamiltonien orienté dans $G$.
  \end{correction}

  On a donc montré que $\textsc{Circuit Hamiltonien} \le_P \textsc{Cycle Hamiltonien}$.
  Comme la réduction se fait en temps polynomial, on a une équivalence de réponse (\textsc{OUI}/\textsc{NON}) entre les deux problèmes.

  Sachant que \textsc{Cycle Hamiltonien} est dans NP et que \textsc{Circuit Hamiltonien} est NP-complet,  
  on en déduit que \textsc{Cycle Hamiltonien} est lui aussi NP-complet.

  \textit{En résumé sur cette feuille :}  
  en partant de \textsc{Sat} (NP-complet par le théorème de Cook–Levin),  
  on a montré successivement que \textsc{3-Sat}, puis \textsc{Circuit Hamiltonien}, puis \textsc{Cycle Hamiltonien}  
  sont NP-complets.  
  Ceci vient donc enrichir notre « catalogue » de problèmes NP-complets.

\end{exercice}

\endgroup
\endinput
