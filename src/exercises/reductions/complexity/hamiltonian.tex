% SPDX-License-Identifier: CC-BY-SA-4.0
% Author: Matthieu Perrin
% Part: Reductions
% Section: Complexity
% Exercise: Hamiltonian circuits

\begingroup

\newcommand\icosian{
  \begin{subfigure}{0.45\textwidth}
    \centering
    \begin{tikzpicture}
      \node[graph node, circle] (11) at (054:05mm) {}; 
      \node[graph node, circle] (13) at (126:05mm) {}; 
      \node[graph node, circle] (15) at (198:05mm) {}; 
      \node[graph node, circle] (17) at (270:05mm) {}; 
      \node[graph node, circle] (19) at (342:05mm) {}; 
      \node[graph node, circle] (20) at (018:12mm) {}; 
      \node[graph node, circle] (21) at (054:12mm) {}; 
      \node[graph node, circle] (22) at (090:12mm) {}; 
      \node[graph node, circle] (23) at (126:12mm) {}; 
      \node[graph node, circle] (24) at (162:12mm) {}; 
      \node[graph node, circle] (25) at (198:12mm) {}; 
      \node[graph node, circle] (26) at (234:12mm) {}; 
      \node[graph node, circle] (27) at (270:12mm) {}; 
      \node[graph node, circle] (28) at (306:12mm) {}; 
      \node[graph node, circle] (29) at (342:12mm) {}; 
      \node[graph node, circle] (30) at (018:20mm) {}; 
      \node[graph node, circle] (32) at (090:20mm) {}; 
      \node[graph node, circle] (34) at (162:20mm) {}; 
      \node[graph node, circle] (36) at (234:20mm) {}; 
      \node[graph node, circle] (38) at (306:20mm) {};

      \draw (11) -- (13) -- (15) -- (17) -- (19) -- (11);
      \draw (20) -- (21) -- (22) -- (23) -- (24) -- (25) -- (26) -- (27) -- (28) -- (29) -- (20);
      \draw (30) -- (32) -- (34) -- (36) -- (38) -- (30);

      \draw (20) -- (30);
      \draw (21) -- (11);
      \draw (22) -- (32);
      \draw (23) -- (13);
      \draw (24) -- (34);
      \draw (25) -- (15);
      \draw (26) -- (36);
      \draw (27) -- (17);
      \draw (28) -- (38);
      \draw (29) -- (19);
    \end{tikzpicture}
    \caption{Plateau de jeu pour le \emph{icosian game}}
    \label{figure:hamiltoncircuit:icosian}
  \end{subfigure}%
}

\newcommand\herschel{
  \begin{subfigure}{0.5\textwidth}
    \centering
    \begin{tikzpicture}[x=8.5mm, y=8.5mm]
      \node[graph node, circle] (44) at (4,4) {}; 
      \node[graph node, circle] (33) at (3,3) {}; 
      \node[graph node, circle] (53) at (5,3) {}; 
      \node[graph node, circle] (02) at (0,2) {}; 
      \node[graph node, circle] (22) at (2,2) {}; 
      \node[graph node, circle] (42) at (4,2) {}; 
      \node[graph node, circle] (62) at (6,2) {}; 
      \node[graph node, circle] (82) at (8,2) {}; 
      \node[graph node, circle] (31) at (3,1) {}; 
      \node[graph node, circle] (51) at (5,1) {}; 
      \node[graph node, circle] (40) at (4,0) {}; 

      \draw (22) -- (31) -- (40);
      \draw (33) -- (42) -- (51);
      \draw (44) -- (53) -- (62);

      \draw (22) -- (33) -- (44);
      \draw (31) -- (42) -- (53);
      \draw (40) -- (51) -- (62);
      
      \draw (02) -- (22);
      \draw (62) -- (82);
      \draw (02) -- (40) -- (82) -- (44) -- (02);
    \end{tikzpicture}
    \caption{Graphe de Herschel}
    \label{figure:hamiltoncircuit:herschel}
  \end{subfigure}
}

\newcommand\reduction{
  \begin{subfigure}{0.49\textwidth}
    \centering
    \begin{tikzpicture}[y=8.5mm, x=9mm]
      \coordinate (left)  at (1,0);
      \coordinate (right) at (9,0);

      \node[graph node, circle] (v1)  at (5,6) {$v_1$};
      \node[graph node, circle] (a11) at (2,5) {$a_1^1$};
      \node[graph node, circle] (b11) at (4,5) {$b_1^1$};
      \node[graph node, circle] (a12) at (6,5) {$a_1^2$};
      \node[graph node, circle] (b12) at (8,5) {$b_1^2$};
      \node[graph node, circle] (v2)  at (5,4) {$v_2$};
      \node[graph node, circle] (a21) at (2,3) {$a_2^1$};
      \node[graph node, circle] (b21) at (4,3) {$b_2^1$};
      \node[graph node, circle] (a22) at (6,3) {$a_2^2$};
      \node[graph node, circle] (b22) at (8,3) {$b_2^2$};
      \node[graph node, circle] (v3)  at (5,2) {$v_3$};
      \node[graph node, circle] (a31) at (2,1) {$a_3^1$};
      \node[graph node, circle] (b31) at (4,1) {$b_3^1$};
      \node[graph node, circle] (a32) at (6,1) {$a_3^2$};
      \node[graph node, circle] (b32) at (8,1) {$b_3^2$};
      
      \path[-latex, ultra thick] (v1) edge node[auto, swap]{$x_1$} (a11);
      \path[-latex            ] (v1) edge (b12);

      \path[-latex, ultra thick] (a11) edge[bend right=7] (b11);
      \path[-latex            ] (b11) edge[bend right=7] (a11);
      \path[-latex, ultra thick] (b11) edge[bend right=7] (a12);
      \path[-latex            ] (a12) edge[bend right=7] (b11);
      \path[-latex, ultra thick] (a12) edge[bend right=7] (b12);
      \path[-latex            ] (b12) edge[bend right=7] (a12);

      \path[-latex            ] (a11) edge (v2);
      \path[-latex, ultra thick] (b12) edge (v2);
      \path[-latex            ] (v2)  edge node[auto, swap]{$x_2$} (a21);
      \path[-latex, ultra thick] (v2)  edge (b22);

      \path[-latex            ] (a21) edge[bend right=7] (b21);
      \path[-latex, ultra thick] (b21) edge[bend right=7] (a21);
      \path[-latex            ] (b21) edge[bend right=7] (a22);
      \path[-latex, ultra thick] (a22) edge[bend right=7] (b21);
      \path[-latex            ] (a22) edge[bend right=7] (b22);
      \path[-latex, ultra thick] (b22) edge[bend right=7] (a22);

      \path[-latex, ultra thick] (a21) edge (v3);
      \path[-latex            ] (b22) edge (v3);
      \path[-latex            ] (v3)  edge node[auto, swap]{$x_3$} (a31);
      \path[-latex, ultra thick] (v3)  edge (b32);

      \path[-latex            ] (a31) edge[bend right=7] (b31);
      \path[-latex, ultra thick] (b31) edge[bend right=7] (a31);
      \path[-latex            ] (b31) edge[bend right=7] (a32);
      \path[-latex, ultra thick] (a32) edge[bend right=7] (b31);
      \path[-latex            ] (a32) edge[bend right=7] (b32);
      \path[-latex, ultra thick] (b32) edge[bend right=7] (a32);

      \draw[-latex, rounded corners=10, ultra thick] (a31) to (left |-a31) to (left |-v1) to (v1);
      \draw[-latex, rounded corners=10            ] (b32) to (right|-b32) to (right|-v1) to (v1);
    \end{tikzpicture}
    \caption{Structure de base de la réduction}
    \label{figure:hamiltoncircuit:reduction}
  \end{subfigure}%
}

\newcommand\gadget{
  \begin{subfigure}{0.49\textwidth}
    \centering
    \begin{tikzpicture}[y=8.5mm, x=9mm]
      \coordinate (left)  at (1,0);
      \coordinate (right) at (9,0);

      \node[graph node, circle] (v1)  at (5,6) {$v_1$};
      \node[graph node, circle] (a11) at (2,5) {$a_1^1$};
      \node[graph node, circle] (b11) at (4,5) {$b_1^1$};
      \node[graph node, circle] (a12) at (6,5) {$a_1^2$};
      \node[graph node, circle] (b12) at (8,5) {$b_1^2$};
      \node[graph node, circle] (v2)  at (5,4) {$v_2$};
      \node[graph node, circle] (a21) at (2,3) {$a_2^1$};
      \node[graph node, circle] (b21) at (4,3) {$b_2^1$};
      \node[graph node, circle] (a22) at (6,3) {$a_2^2$};
      \node[graph node, circle] (b22) at (8,3) {$b_2^2$};
      \node[graph node, circle] (v3)  at (5,2) {$v_3$};
      \node[graph node, circle] (a31) at (2,1) {$a_3^1$};
      \node[graph node, circle] (b31) at (4,1) {$b_3^1$};
      \node[graph node, circle] (a32) at (6,1) {$a_3^2$};
      \node[graph node, circle] (b32) at (8,1) {$b_3^2$};

      \node[graph node, circle, fill=black!30] (C1)  at (3,4) {$C^1$};
      \node[graph node, circle, fill=black!30] (C2)  at (7,2) {$C^2$};
      
      \path[-latex] (v1) edge (a11);
      \path[-latex] (v1) edge (b12);

      \path[-latex] (a11) edge[bend right=7] (b11);
      \path[-latex] (b11) edge[bend right=7] (a11);
      \path[-latex] (b11) edge[bend right=7] (a12);
      \path[-latex] (a12) edge[bend right=7] (b11);
      \path[-latex] (a12) edge[bend right=7] (b12);
      \path[-latex] (b12) edge[bend right=7] (a12);

      \path[-latex] (a11) edge (v2);
      \path[-latex] (b12) edge (v2);
      \path[-latex] (v2)  edge (a21);
      \path[-latex] (v2)  edge (b22);

      \path[-latex] (a21) edge[bend right=7] (b21);
      \path[-latex] (b21) edge[bend right=7] (a21);
      \path[-latex] (b21) edge[bend right=7] (a22);
      \path[-latex] (a22) edge[bend right=7] (b21);
      \path[-latex] (a22) edge[bend right=7] (b22);
      \path[-latex] (b22) edge[bend right=7] (a22);

      \path[-latex] (a21) edge (v3);
      \path[-latex] (b22) edge (v3);
      \path[-latex] (v3)  edge (a31);
      \path[-latex] (v3)  edge (b32);

      \path[-latex] (a31) edge[bend right=7] (b31);
      \path[-latex] (b31) edge[bend right=7] (a31);
      \path[-latex] (b31) edge[bend right=7] (a32);
      \path[-latex] (a32) edge[bend right=7] (b31);
      \path[-latex] (a32) edge[bend right=7] (b32);
      \path[-latex] (b32) edge[bend right=7] (a32);

      \path[-latex] (a11) edge               (C1);  \path[-latex] (C1)  edge (b11);
      \path[-latex] (a22) edge               (C2);  \path[-latex] (C2)  edge (b22);
      \path[-latex] (b32) edge               (C2);  \path[-latex] (C2)  edge (a32);

      
      \draw[-latex, rounded corners=10] (a31) to (left |-a31) to (left |-v1) to (v1);
      \draw[-latex, rounded corners=10] (b32) to (right|-b32) to (right|-v1) to (v1);
    \end{tikzpicture}
    \caption{Instrumentation par gadgets (portes de clauses)}
    \label{figure:hamiltoncircuit:gadgets}
  \end{subfigure}%
}

\begin{exercice}[\textsc{Circuit Hamiltonien} est NP-complet%
    \footnote{Vidéo détaillée de la réduction de \textsc{3-SAT} à \textsc{Cycle Hamiltonien} : \url{https://www.youtube.com/watch?v=OHnX-R_SBpM}}]
  \label{exo:reductions/complexity/hamiltonian}

  On commence par introduire quelques définitions sur les graphes.
  \begin{itemize}
  \item Un \emph{graphe orienté} est un couple $G = \langle S, A \rangle$,
    où $S$ est l'ensemble de ses \emph{sommets}
    et $A \subseteq S \times S$ est l'ensemble de ses \emph{arcs}.
  \item Un \emph{chemin} de $G$ est une suite finie de sommets $(s_1, s_2, \dots, s_n)$
    tels que pour tout $1 \le i < n$, $\langle s_i, s_{i+1} \rangle \in A$.
  \item Un \emph{circuit} de $G$ est un chemin qui revient à son point de départ, c'est-à-dire tel que $s_1 = s_n$.
  \item Un circuit de $G$ est dit \emph{hamiltonien} s'il passe \emph{une et une seule fois} par chaque sommet de $G$,
    c'est-à-dire si pour tout $s\in S$, il existe un unique $i \in \{1, \dots, n-1\}$ tel que $s_i = s$.
  \item Un graphe $G$ est dit \emph{hamiltonien} s'il existe un \emph{circuit hamiltonien} dans $G$.
  \end{itemize}

  On considère maintenant le problème \textsc{Circuit Hamiltonien} défini ci-dessous.
  Le but de cet exercice est de montrer que \textsc{Circuit Hamiltonien} est NP-complet.

  \Probleme{Circuit Hamiltonien}{
    Un graphe orienté $G$.
  }{
    $G$ est-il hamiltonien ?
  }

  \begin{question}
  \item On considère les graphes des figures \ref{figure:hamiltoncircuit:icosian} et \ref{figure:hamiltoncircuit:herschel}.
    Ces deux graphes sont-ils hamiltoniens ?
    Justifier en donnant un circuit hamiltonien, ou en expliquant brièvement pourquoi il ne peut pas y en avoir.
  \end{question}

  \begin{figure}[h]%
    \centering
    \icosian%
    \herschel%
  \end{figure}

  \ifcorrection{\newpage}

  \begin{correction}
    Le graphe de la figure~\ref{figure:hamiltoncircuit:icosian} est hamiltonien. On a par exemple le circuit suivant.
    \begin{center}
      \begin{tikzpicture}
        \node[graph node, circle, inner sep=1.5pt] (11) at (054:05mm) {}; 
        \node[graph node, circle, inner sep=1.5pt] (13) at (126:05mm) {}; 
        \node[graph node, circle, inner sep=1.5pt] (15) at (198:05mm) {}; 
        \node[graph node, circle, inner sep=1.5pt] (17) at (270:05mm) {}; 
        \node[graph node, circle, inner sep=1.5pt] (19) at (342:05mm) {}; 
        \node[graph node, circle, inner sep=1.5pt] (20) at (018:12mm) {}; 
        \node[graph node, circle, inner sep=1.5pt] (21) at (054:12mm) {}; 
        \node[graph node, circle, inner sep=1.5pt] (22) at (090:12mm) {}; 
        \node[graph node, circle, inner sep=1.5pt] (23) at (126:12mm) {}; 
        \node[graph node, circle, inner sep=1.5pt] (24) at (162:12mm) {}; 
        \node[graph node, circle, inner sep=1.5pt] (25) at (198:12mm) {}; 
        \node[graph node, circle, inner sep=1.5pt] (26) at (234:12mm) {}; 
        \node[graph node, circle, inner sep=1.5pt] (27) at (270:12mm) {}; 
        \node[graph node, circle, inner sep=1.5pt] (28) at (306:12mm) {}; 
        \node[graph node, circle, inner sep=1.5pt] (29) at (342:12mm) {}; 
        \node[graph node, circle, inner sep=1.5pt] (30) at (018:20mm) {}; 
        \node[graph node, circle, inner sep=1.5pt] (32) at (090:20mm) {}; 
        \node[graph node, circle, inner sep=1.5pt] (34) at (162:20mm) {}; 
        \node[graph node, circle, inner sep=1.5pt] (36) at (234:20mm) {}; 
        \node[graph node, circle, inner sep=1.5pt] (38) at (306:20mm) {};

        \draw[densely dotted] (17) -- (19);
        \draw[densely dotted] (26) -- (27);
        \draw[densely dotted] (28) -- (29);
        \draw[densely dotted] (36) -- (38);

        \draw[densely dotted] (20) -- (30);
        \draw[densely dotted] (21) -- (11);
        \draw[densely dotted] (22) -- (32);
        \draw[densely dotted] (23) -- (13);
        \draw[densely dotted] (24) -- (34);
        \draw[densely dotted] (25) -- (15);
        
        \draw (38) -- (30) -- (32) -- (34) -- (36)
        -- (26) -- (25) -- (24) -- (23) -- (22) -- (21) -- (20) -- (29)
        -- (19) -- (11)-- (13) -- (15) -- (17)
        -- (27) -- (28)
        -- (38);
      \end{tikzpicture}
    \end{center}
    
    Le graphe de la figure~\ref{figure:hamiltoncircuit:herschel} n'est pas hamiltonien. Pour s'en convaincre, on peut tenter de construire un circuit hamiltonien,
    en choisissant un arc d'entrée et un arc de sortie pour le sommet central (en rouge), ce qui donne trois cas possibles,
    à symétrie près. Cela donne des contraintes sur les sommets adjacents (en bleu), puis sur les deux sommets extérieurs (en violet),
    et on arrive rapidement à une contradiction. 

    \begin{center}
      \begin{tikzpicture}[x=5mm, y=5mm]
        \node[graph node, circle, fill=green , inner sep=2pt] (44) at (4,4) {}; 
        \node[graph node, circle, fill=blue  , inner sep=2pt] (33) at (3,3) {}; 
        \node[graph node, circle, fill=blue  , inner sep=2pt] (53) at (5,3) {}; 
        \node[graph node, circle, fill=violet, inner sep=2pt] (02) at (0,2) {}; 
        \node[graph node, circle, fill=green , inner sep=2pt] (22) at (2,2) {}; 
        \node[graph node, circle, fill=red   , inner sep=2pt] (42) at (4,2) {}; 
        \node[graph node, circle, fill=green , inner sep=2pt] (62) at (6,2) {}; 
        \node[graph node, circle, fill=violet, inner sep=2pt] (82) at (8,2) {}; 
        \node[graph node, circle, fill=blue  , inner sep=2pt] (31) at (3,1) {}; 
        \node[graph node, circle, fill=blue  , inner sep=2pt] (51) at (5,1) {}; 
        \node[graph node, circle, fill=green , inner sep=2pt] (40) at (4,0) {}; 

        \draw[blue]           (22) to (31) to (40);
        \draw[red]            (33) to (42) to (51);
        \draw[blue]           (44) to (53) to (62);

        \draw[blue]           (22) to (33);
        \draw[densely dotted] (33) to (44);
        \draw[densely dotted] (31) to (42) to (53);
        \draw                 (40) to (51) to (62);
        
        \draw[densely dotted] (02) to (22);
        \draw                 (62) to (82);
        \draw[violet]         (44) to (02) to (40);
        \draw (40) to (82) to (44) ;
      \end{tikzpicture}
      \quad
      \begin{tikzpicture}[x=5mm, y=5mm]
        \node[graph node, circle, fill=green , inner sep=2pt] (44) at (4,4) {}; 
        \node[graph node, circle, fill=blue  , inner sep=2pt] (33) at (3,3) {}; 
        \node[graph node, circle, fill=blue  , inner sep=2pt] (53) at (5,3) {}; 
        \node[graph node, circle, fill=violet, inner sep=2pt] (02) at (0,2) {}; 
        \node[graph node, circle, fill=green , inner sep=2pt] (22) at (2,2) {}; 
        \node[graph node, circle, fill=red   , inner sep=2pt] (42) at (4,2) {}; 
        \node[graph node, circle, fill=green , inner sep=2pt] (62) at (6,2) {}; 
        \node[graph node, circle, fill=violet, inner sep=2pt] (82) at (8,2) {}; 
        \node[graph node, circle, fill=blue  , inner sep=2pt] (31) at (3,1) {}; 
        \node[graph node, circle, fill=blue  , inner sep=2pt] (51) at (5,1) {}; 
        \node[graph node, circle, fill=green , inner sep=2pt] (40) at (4,0) {}; 
        
        \draw[blue]           (22) to (31) to (40);
        \draw[red]            (33) to (42);
        \draw[densely dotted] (42) to (51);
        \draw[densely dotted] (44) to (53) to (62);
        
        \draw[densely dotted] (22) to (33) to (44);
        \draw[densely dotted] (31) to (42);
        \draw[red]            (42) to (53);
        \draw[blue]           (40) to (51) to (62);
        
        \draw[violet]         (02) to (22);
        \draw[violet]         (62) to (82);
        \draw[densely dotted] (02) to (40) to (82);
        \draw[violet]         (82) to (44) to (02);
      \end{tikzpicture}
      \quad
      \begin{tikzpicture}[x=5mm, y=5mm]
        \node[graph node, circle, fill=green , inner sep=2pt] (44) at (4,4) {}; 
        \node[graph node, circle, fill=blue  , inner sep=2pt] (33) at (3,3) {}; 
        \node[graph node, circle, fill=blue  , inner sep=2pt] (53) at (5,3) {}; 
        \node[graph node, circle, fill=violet, inner sep=2pt] (02) at (0,2) {}; 
        \node[graph node, circle, fill=green , inner sep=2pt] (22) at (2,2) {}; 
        \node[graph node, circle, fill=red   , inner sep=2pt] (42) at (4,2) {}; 
        \node[graph node, circle, fill=green , inner sep=2pt] (62) at (6,2) {}; 
        \node[graph node, circle, fill=violet, inner sep=2pt] (82) at (8,2) {}; 
        \node[graph node, circle, fill=blue  , inner sep=2pt] (31) at (3,1) {}; 
        \node[graph node, circle, fill=blue  , inner sep=2pt] (51) at (5,1) {}; 
        \node[graph node, circle, fill=green , inner sep=2pt] (40) at (4,0) {}; 
        
        \draw[blue]           (22) to (31) to (40);
        \draw[densely dotted] (33) to (42);
        \draw[red]            (42) to (51);
        \draw                 (44) to (53) to (62);
        
        \draw[blue]           (22) to (33) to (44);
        \draw[densely dotted] (31) to (42);
        \draw[red]            (42) to (53);
        \draw                 (40) to (51) to (62);
        
        \draw[densely dotted] (02) to (22);
        \draw                 (62) to (82);
        \draw[violet]         (44) to (02) to (40);
        \draw[densely dotted] (40) to (82) to (44) ;
      \end{tikzpicture}
    \end{center}
  \end{correction}

  
  \begin{question}
  \item On cherche maintenant à montrer que \textsc{Circuit Hamiltonien} est dans NP.
    Quelle forme peut prendre un \emph{certificat} pour $G \in \textsc{Circuit Hamiltonien}$ ?
  \end{question}
  \begin{correction}
    Un certificat naturel est une \emph{liste ordonnée} $c = [c[1], c[2], \dots, c[n]]$ des sommets telle que le chemin
    $$(c[1], c[2], \dots, c[n], c[1])$$
    est un circuit hamiltonien de $G$. 
  \end{correction}

  \begin{question}
  \item Soit $G$ un graphe hamiltonien. Montrer que la \emph{taille} d'un tel certificat pour $G$
    est polynomiale par rapport à la taille de l'entrée $G$.
  \end{question}
  \begin{correction}
    En encodant un sommet par son identifiant, la taille du certificat est $n$ identifiants,
    donc $\mathcal{O}(n\log n)$ bits si les identifiants sont sur $\mathcal{O}(\log n)$ bits.
    Par rapport à une entrée encodée en listes d'adjacence (taille $\Theta(|S| + |A|)$)
    ou matrice d'adjacence (taille $\Theta(|S|^2)$), la taille du certificat est \emph{polynomiale}.
  \end{correction}

  \ifcorrection{\newpage}
  \begin{question}
  \item Proposer un \emph{algorithme vérificateur} qui, étant donné en entrée un graphe $G$ et un candidat $c$,
    décide si $c$ est un certificat \emph{valide} pour l'appartenance de $G$ à \textsc{Circuit Hamiltonien}.
  \end{question}
  \begin{correction}
    On a un algorithme simple en $\mathcal{O}(n^3)$. \\
    \begin{algorithm}[H]
      \SetKwFunction{Verify}{verify}
      \SetKwFunction{Contains}{contains}
      \SetKwFunction{Length}{length}
      \Fun{$\Verify(G=\langle S, A \rangle, c) $}{
        \tcp{vérification de $n = |S|$ en $\mathcal{O}(1)$}
        \lIf{$|c| \neq |S|$}{
          \Return \False;
        }
        \tcp{vérification que tous les sommets du chemin sont dans $G$ en $\mathcal{O}(n)$}
        \For{$i$ \From $1$ \To $n$}{
          \lIf{$\lnot \Contains(S, c[i])$}{
            \Return \False;
          }
        }
        
        \tcp{vérification de l'unicité des $c[i]$ en $\mathcal{O}(n^2)$}
        \For{$i$ \From $1$ \To $n$}{
          \For{$j$ \From $1$ \To $n$}{
            \lIf{$i \neq j \land c[i] = c[j]$}{
              \Return \False;
            }
          }
        }
        \tcp{vérification que les arcs sont bien présents en $\mathcal{O}(n \times m) = \mathcal{O}(n^3)$}
        \For{$i$ \From $1$ \To $n$}{
          \lIf{$\lnot \Contains(A, \langle c[i], c[i \mod n + 1] \rangle)$}{
            \Return \False;
          }
        }
        \Return \True;
      }
      \tcp{Cherche un élément dans un tableau en $\mathcal{O}(|tab|)$}
      \Fun{$\Contains(\mathit{tab}, \mathit{element}) $}{
        \For{$i$ \From $1$ \To $\mathit{tab}.\Length()$}{
          \lIf{$\mathit{tab}[i] = \mathit{element}$}{
            \Return \True;
          }
        }
        \Return \False;
      }
    \end{algorithm}
  \end{correction}

  \begin{question}
  \item Calculer la \emph{complexité en temps} dans le pire cas de votre vérificateur.
  \end{question}
  \begin{correction}
    On a un algorithme en $\mathcal{O}(n^3)$, où $n$ est inférieur à la taille des entrées (puisque $c$ est en $\mathcal{O}(n \log(n))$),
    donc en temps \emph{polynomial} en la taille de l'entrée.
  \end{correction}

  \begin{question}
  \item Conclure que \textsc{Circuit Hamiltonien} est dans NP.
  \end{question}
  \begin{correction}
    Il existe un certificat de taille polynomiale (une permutation des sommets) et un vérificateur en temps polynomial.  
    Par définition, $\textsc{Circuit Hamiltonien} \in \mathbf{NP}$.
  \end{correction}

  \ifnotcorrection{\pagebreak}
  On cherche maintenant à démontrer que le problème \textsc{Circuit Hamiltonien} est NP-dur par une réduction polynomiale de \textsc{Sat} vers \textsc{Circuit Hamiltonien}.
  On rappelle :

  \Probleme{Sat}{
    Une formule booléenne $\varphi$ en forme normale conjonctive (FNC), construite sur un ensemble $X=\{x_1,\dots,x_n\}$ de variables.
  }{
    Existe-t-il une affectation (Vrai/Faux) de chaque variable $x_i$ qui rend $\varphi$ satisfiable ?
  }

  On considère les circuits \emph{à rotation près} : par exemple $(1,2,3,1)$ et $(2,3,1,2)$ représentent le même circuit.

  \ifcorrection{\newpage}
  \begin{question}
  \item Dans le graphe de la figure~\ref{figure:hamiltoncircuit:reduction}, combien existe-t-il de circuits hamiltoniens (à rotation près) ? 
  \end{question}
  \begin{correction}
    Par rotation, on peut choisir de commencer par le sommet $v_1$. Le schéma impose, pour chaque $i\in\{1,2,3\}$,
    un choix binaire au sommet $v_i$ : emprunter l'arc $\langle v_i,a_i^1\rangle$ ou l'arc $\langle v_i, b_i^2\rangle$.  
    Ces choix sont indépendants et, pour chacun, il n'y a qu'une seule façon d'inclure tous les sommets $a_i^j$ et $b_i^j$. 
    \emph{À rotation près}, on obtient donc exactement $2^3=8$ circuits hamiltoniens,
    correspondant à tous les choix possibles de sortie des trois $v_i$.
  \end{correction}
  
  Pour $i \in \{1,2,3\}$, on note $x_i$ le prédicat unaire sur les circuits $c$ défini par :
  $x_i(c) \eqdef \text{$c$ emprunte l'arc}\langle v_i, a_i^1\rangle$.

  \begin{question}
  \item Caractériser de manière \emph{unique} le circuit hamiltonien $c_0$ dessiné en gras sur la figure~\ref{figure:hamiltoncircuit:reduction}
    par une formule booléenne en $x_1(c_0)$, $x_2(c_0)$ et $x_3(c_0)$.
  \end{question}

  \begin{figure}[h!]%
    \centering
    \reduction%
    \gadget%
  \end{figure}

  \begin{correction}
    Dans le circuit en gras, un choix est imposé pour chaque $x_i(c_0)$ :
    $x_1(c_0)=\text{Vrai}$, $x_2(c_0)=\text{Faux}$ et $x_3(c_0)=\text{Faux}$.
    C'est donc l'unique circuit hamiltonien \emph{à rotation près} du graphe tel que :
    $$ x_1 \land \lnot x_2 \land \lnot x_3.$$
  \end{correction}
  
  \begin{question}
  \item On ajoute un nouveau sommet $C^1$ et deux arcs $\langle a_1^1, C^1\rangle$ et $\langle C^1, b_1^1\rangle$
    comme à gauche de la figure~\ref{figure:hamiltoncircuit:gadgets}.
    Définir un prédicat logique $P(c)$ sur les circuits hamiltoniens $c$,
    exprimé en fonction de $x_1(c)$, $x_2(c)$ et $x_3(c)$,
    qui caractérise l'ensemble des circuits hamiltoniens du graphe \emph{modifié}. 
  \end{question}
  \begin{correction}
    Tout circuit hamiltonien doit désormais \emph{nécessairement} passer par $C^1$.
    Or $C^1$ n'est accessible que si le parcours a pris la branche gauche au niveau de $v_1$ (celle qui mène à $a_1^1$ puis vers $b_1^1$).  
    Par conséquent, l’ensemble des circuits hamiltoniens du graphe \emph{modifié} est exactement $\{c \mid P(c) \}$, où $P(c)  \eqdef x_1(c)$. 
  \end{correction}

  \begin{question}
  \item Caractériser tous les circuits hamiltoniens du graphe de la figure~\ref{figure:hamiltoncircuit:gadgets}. 
  \end{question}
  \begin{correction}
    On a deux gadgets ($C^1$ et $C^2$), donc le prédicat est une conjonction des deux clauses : 
    \begin{itemize}
    \item $C_1$ impose seulement $x_1(c)$ ;
    \item $C^2$ peut être accédé soit par le chemin $a_2^2 \to C^2 \to b_2^2$, soit par le chemin $b_3^2 \to C^2 \to a_3^2$,
      et impose donc la disjonction $x_2(c) \lor \lnot x_3(c)$. 
    \end{itemize}
    L'ensemble des chemins hamiltoniens de ce graphe est donc $\{ c \mid x_1(c) \land (x_2(c) \lor \lnot x_3(c))\}$.
  \end{correction}

  \ifcorrection{\newpage}
  \begin{question}
  \item En modifiant localement la figure~\ref{figure:hamiltoncircuit:reduction} sur le même principe,
    proposer des graphes dont l'ensemble des circuits hamiltoniens est exactement :
    \begin{center}
      $\{c \mid \lnot x_2(c)\}$ \quad\quad $\{c \mid x_1(c) \land \lnot x_2(c)\}$  \quad\quad $\{c \mid x_1(c) \lor \lnot x_2(c)\}$ 
    \end{center}
  \end{question}
  \begin{correction}
    \begin{itemize}
    \item $\{ c \mid \lnot x_2(c) \}$ : insérer un sommet $C^2$ et deux arcs $\langle b_2^2, C^2\rangle$ et $\langle C^2, a_2^2 \rangle$
      ($C_2$ est entre $b_2^2$ et $a_2^2$, accessible de la droite vers la gauche)
    \item $\{ c \mid x_1(c) \land \lnot x_2(c) \}$ : combiner les deux portes précédentes
      (un sommet $C^1$ entre $a_1^1$ et $b_1^1$ et un sommet $C^2$ entre $b_2^2$ et $a_2^2$). 
    \item $\{ c \mid x_1(c) \lor \lnot x_2(c) \}$ : il faut utiliser le même sommet dans les deux portes.
      On ajoute un sommet $C$ et quatre arcs : $\langle a_1^1, C\rangle$, $\langle C, b_1^1 \rangle$, $\langle b_2^2, C\rangle$ et $\langle C, a_2^2 \rangle$.
    \end{itemize}
  \end{correction}
  
  \begin{question}
  \item Construire un graphe orienté dont les circuits hamiltoniens sont exactement les circuits $c$ tels que :
    $$(\lnot x_1(c) \lor \lnot x_2(c) \lor x_4(c)) \land (x_1(c) \lor x_2(c)) \land (\lnot x_2(c) \lor x_3(c) \lor \lnot x_4(c))$$
  \end{question}
  \begin{correction}
    Attention, la formule $\varphi$ a trois clauses et quatre variables, donc il faut agrandir le graphe de base pour avoir
    quatre gadgets de variables contenant chacun trois sommets $a_i^j$ et trois sommets $b_i^j$. Voir le graphe sur la page suivante.

    \vspace{-2mm}
    \begin{center}
      \begin{tikzpicture}[y=13mm, x=8mm]
        \coordinate (left)  at ( 1,0);
        \coordinate (right) at (13,0);

        \node[graph node, circle] (v1)  at ( 7,8) {$v_1$};
        \node[graph node, circle] (a11) at ( 2,7) {$a_1^1$};
        \node[graph node, circle] (b11) at ( 4,7) {$b_1^1$};
        \node[graph node, circle] (a12) at ( 6,7) {$a_1^2$};
        \node[graph node, circle] (b12) at ( 8,7) {$b_1^2$};
        \node[graph node, circle] (a13) at (10,7) {$a_1^3$};
        \node[graph node, circle] (b13) at (12,7) {$b_1^3$};
        \node[graph node, circle] (v2)  at ( 7,6) {$v_2$};
        \node[graph node, circle] (a21) at ( 2,5) {$a_2^1$};
        \node[graph node, circle] (b21) at ( 4,5) {$b_2^1$};
        \node[graph node, circle] (a22) at ( 6,5) {$a_2^2$};
        \node[graph node, circle] (b22) at ( 8,5) {$b_2^2$};
        \node[graph node, circle] (a23) at (10,5) {$a_2^3$};
        \node[graph node, circle] (b23) at (12,5) {$b_2^3$};
        \node[graph node, circle] (v3)  at ( 7,4) {$v_3$};
        \node[graph node, circle] (a31) at ( 2,3) {$a_3^1$};
        \node[graph node, circle] (b31) at ( 4,3) {$b_3^1$};
        \node[graph node, circle] (a32) at ( 6,3) {$a_3^2$};
        \node[graph node, circle] (b32) at ( 8,3) {$b_3^2$};
        \node[graph node, circle] (a33) at (10,3) {$a_3^3$};
        \node[graph node, circle] (b33) at (12,3) {$b_3^3$};
        \node[graph node, circle] (v4)  at ( 7,2) {$v_4$};
        \node[graph node, circle] (a41) at ( 2,1) {$a_4^1$};
        \node[graph node, circle] (b41) at ( 4,1) {$b_4^1$};
        \node[graph node, circle] (a42) at ( 6,1) {$a_4^2$};
        \node[graph node, circle] (b42) at ( 8,1) {$b_4^2$};
        \node[graph node, circle] (a43) at (10,1) {$a_4^3$};
        \node[graph node, circle] (b43) at (12,1) {$b_4^3$};

        \node[graph node, fill=black!30] (C1) at ( -.7,4) {$\lnot x_1 \land \lnot x_2 \land x_4$};
        \path[-latex] (b11) edge (C1);
        \path[-latex] (C1)  edge (a11);
        \path[-latex] (b21) edge (C1);
        \path[-latex] (C1)  edge (a21);
        \path[-latex] (a41) edge (C1);
        \path[-latex] (C1)  edge (b41);

        \node[graph node, fill=black!30] (C2) at (14.7,6) {$x_1 \land x_2$};
        \path[-latex] (a12) edge[bend right=7] (C2);
        \path[-latex] (C2)  edge[bend left=7 ] (b12);
        \path[-latex] (a22) edge[bend left=7 ] (C2);
        \path[-latex] (C2)  edge[bend right=7] (b22);

        \node[graph node, fill=black!30] (C3) at (14.7,2) {$\lnot x_2 \land x_3 \land \lnot x_4$};
        \path[-latex] (b23) edge (C3);
        \path[-latex] (C3)  edge (a23);
        \path[-latex] (a33) edge (C3);
        \path[-latex] (C3)  edge (b33);
        \path[-latex] (b43) edge (C3);
        \path[-latex] (C3)  edge (a43);
        
        \path[-latex] (v1) edge[bend right=7] (a11);
        \path[-latex] (v1) edge[bend left =7] (b13);

        \path[-latex] (a11) edge[bend right=7] (b11);
        \path[-latex] (b11) edge[bend right=7] (a11);
        \path[-latex] (b11) edge[bend right=7] (a12);
        \path[-latex] (a12) edge[bend right=7] (b11);
        \path[-latex] (a12) edge[bend right=7] (b12);
        \path[-latex] (b12) edge[bend right=7] (a12);
        \path[-latex] (b12) edge[bend right=7] (a13);
        \path[-latex] (a13) edge[bend right=7] (b12);
        \path[-latex] (a13) edge[bend right=7] (b13);
        \path[-latex] (b13) edge[bend right=7] (a13);

        \path[-latex] (a11) edge[bend right=7] (v2);
        \path[-latex] (b13) edge[bend left =7] (v2);
        \path[-latex] (v2)  edge[bend right=7] (a21);
        \path[-latex] (v2)  edge[bend left =7] (b23);

        \path[-latex] (a21) edge[bend right=7] (b21);
        \path[-latex] (b21) edge[bend right=7] (a21);
        \path[-latex] (b21) edge[bend right=7] (a22);
        \path[-latex] (a22) edge[bend right=7] (b21);
        \path[-latex] (a22) edge[bend right=7] (b22);
        \path[-latex] (b22) edge[bend right=7] (a22);
        \path[-latex] (b22) edge[bend right=7] (a23);
        \path[-latex] (a23) edge[bend right=7] (b22);
        \path[-latex] (a23) edge[bend right=7] (b23);
        \path[-latex] (b23) edge[bend right=7] (a23);

        \path[-latex] (a21) edge[bend right=7] (v3);
        \path[-latex] (b23) edge[bend left =7] (v3);
        \path[-latex] (v3)  edge[bend right=7] (a31);
        \path[-latex] (v3)  edge[bend left =7] (b33);

        \path[-latex] (a31) edge[bend right=7] (b31);
        \path[-latex] (b31) edge[bend right=7] (a31);
        \path[-latex] (b31) edge[bend right=7] (a32);
        \path[-latex] (a32) edge[bend right=7] (b31);
        \path[-latex] (a32) edge[bend right=7] (b32);
        \path[-latex] (b32) edge[bend right=7] (a32);
        \path[-latex] (b32) edge[bend right=7] (a33);
        \path[-latex] (a33) edge[bend right=7] (b32);
        \path[-latex] (a33) edge[bend right=7] (b33);
        \path[-latex] (b33) edge[bend right=7] (a33);

        \path[-latex] (a31) edge[bend right=7] (v4);
        \path[-latex] (b33) edge[bend left =7] (v4);
        \path[-latex] (v4)  edge[bend right=7] (a41);
        \path[-latex] (v4)  edge[bend left =7] (b43);

        \path[-latex] (a41) edge[bend right=7] (b41);
        \path[-latex] (b41) edge[bend right=7] (a41);
        \path[-latex] (b41) edge[bend right=7] (a42);
        \path[-latex] (a42) edge[bend right=7] (b41);
        \path[-latex] (a42) edge[bend right=7] (b42);
        \path[-latex] (b42) edge[bend right=7] (a42);
        \path[-latex] (b42) edge[bend right=7] (a43);
        \path[-latex] (a43) edge[bend right=7] (b42);
        \path[-latex] (a43) edge[bend right=7] (b43);
        \path[-latex] (b43) edge[bend right=7] (a43);
        
        \draw[-latex, rounded corners=10] (a41) to (left |-a41) to (left |-v1) to (v1);
        \draw[-latex, rounded corners=10] (b43) to (right|-b43) to (right|-v1) to (v1);
      \end{tikzpicture}
    \end{center}
  \end{correction}
  
  \ifcorrection{\newpage}
  \begin{question}
  \item Définir une transformation $\mathcal{G}$ qui, à toute formule $\varphi$
    en forme normale conjonctive sur des variables $x_1,\dots,x_n$, associe un graphe orienté $\mathcal{G}(\varphi)$
    dont $\varphi$ caractérise les circuits hamiltoniens.
    En déduire que $\varphi \in \textsc{Sat} \iff \mathcal{G}(\varphi) \in \textsc{Circuit Hamiltonien}$.
  \end{question}
  \begin{correction}
    $$
    \begin{array}{rcl@{\quad}l}
      \mathcal{G}(\varphi) &=& \langle S(\varphi), A(\varphi) \rangle, \text{ avec :} \\
      S(\varphi) &=&  \{ v_i   \mid 1 \le i \le n  \}                                                                                                     & \le n\\
      &\cup& \{ a_i^j \mid 1 \le i \le n \land 1 \le j \le k\}                                                                                            & \le n \times k\\
      &\cup& \{ b_i^j \mid 1 \le i \le n \land 1 \le j \le k\}                                                                                            & \le n \times k\\
      &\cup& \{ C^j   \mid 1 \le j \le k\}                                                                                                                 & \le k\\ 
      A(\varphi) &=&  \{ \langle v_i, a_i^1 \rangle, \langle v_i, b_i^k  \rangle             \mid 1 \le i \le n  \}                                        & \le n\\
      &\cup& \{ \langle a_i^1, v_{i \bmod n + 1} \rangle, \langle b_i^k, v_{i \bmod n + 1} \rangle \mid 1 \le i \le n  \}                                        & \le n\\
      &\cup& \{ \langle a_i^j, b_i^j \rangle,          \langle b_i^j, a_i^j \rangle          \mid 1 \le i \le n  \land 1 \le j \le k \}                    & \le n \times k\\
      &\cup& \{ \langle b_i^j, a_i^{j+1} \rangle,       \langle a_i^{j+1}, b_i^j \rangle       \mid 1 \le i \le n  \land 1 \le j < k \}                      & \le n \times k\\
      &\cup& \{ \langle a_i^j, C^j \rangle,            \langle C^j, b_i^j \rangle            \mid x_i \text{ est un litéral de la clause } C^j \}          & \le n \times k\\
      &\cup& \{ \langle b_i^j, C^j \rangle,            \langle C^j, a_i^j \rangle            \mid \lnot x_i \text{ est un litéral de la clause } C^j \}    & \le n \times k
    \end{array}
    $$
    La formule $\varphi$ décrit précisément les circuits hamiltoniens de $\mathcal{G}(\varphi)$.
    Il existe donc une affectation des variables de $\varphi$ si, et seulement si,
    il existe un circuits hamiltoniens dans $\mathcal{G}(\varphi)$.
    Autrement dit, $$\varphi \in \textsc{Sat} \iff \mathcal{G}(\varphi) \in \textsc{Circuit Hamiltonien}.$$
  \end{correction}
  
  \begin{question}
  \item Soit $\varphi$ une formule à $n$ variables, et $k$ clauses.
    \begin{itemize}
    \item donner une surapproximation du nombre de sommets et d'arcs de $\mathcal{G}(\varphi)$ en fonction de $n$ et $k$ ;
    \item conclure que la construction $\mathcal{G}$ s'effectue en temps polynomial.
    \end{itemize}
  \end{question}
  \begin{correction}
    Le calcul est détaillé sur la transformation de la question précédente.
    Le nombre de sommets dans $\mathcal{G}(\varphi)$ est borné par $2 n k + n + k = \mathcal{O}(n k)$.
    Le nombre d'arcs dans $\mathcal{G}(\varphi)$ est borné par $4 n k + 2 n = \mathcal{O}(n k)$.
    La transformation n'a pas de difficulté algorithmique, donc elle s'effectue en temps polynomial.
  \end{correction}
  
  \begin{question}
  \item Que peut-on conclure sur la complexité de \textsc{Circuit Hamiltonien} ? 
  \end{question}
  \begin{correction}
    On a une réduction polynomiale de \textsc{Sat} vers \textsc{Circuit Hamiltonien}.
    On en déduit que \textsc{Circuit Hamiltonien} est NP-difficile.
    De plus, on sait que \textsc{Circuit Hamiltonien} est NP.
    On en déduit que \textsc{Circuit Hamiltonien} est NP-complet.
  \end{correction}

\end{exercice}

\endgroup
\endinput
