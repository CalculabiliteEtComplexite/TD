% SPDX-License-Identifier: CC-BY-SA-4.0
% Author: Matthieu Perrin
% Part: Turing machines
% Section: Deterministic Turing machines
% Exercise: Conception of Turing machines

\begingroup

\begin{exercice}[Conception de machines de Turing]\label{exo:turingmachines/definitions/languages}

  Dans cet exercice, on se limitera au modèle strict des machines de Turing sur un seul ruban. 

  \begin{question}
  \item Construire une machine de Turing déterministe qui décide le langage $\{ w c w \mid w\in \{a, b\}^\star \}$.
  \end{question}
  \begin{correction}
    \begin{tikzpicture}[turingMachine, baseline=(0)]
      \state[initial above] (0) at (2,1) {$0$}; 
      \state                (1) at (3,2) {$1$}; 
      \state                (2) at (4,2) {$2$}; 
      \state                (3) at (3,0) {$3$}; 
      \state                (4) at (4,0) {$4$}; 
      \state                (5) at (5,1) {$5$}; 
      \state                (6) at (1,1) {$6$}; 
      \state[accepting]     (7) at (0,1) {$7$}; 

      \path (0) edge  node       {\smTMtransR{a}{\blank}}      (1);
      \path (1) edge  node       {\smTMtransR{c}{c}}           (2);
      \path (2) edge  node       {\smTMtransL{a}{X}}           (5);
      \path (0) edge  node[swap] {\smTMtransR{b}{\blank}}      (3);
      \path (3) edge  node[swap] {\smTMtransR{c}{c}}           (4);
      \path (4) edge  node[swap] {\smTMtransL{b}{X}}           (5);
      \path (5) edge  node[swap] {\smTMtransR{\blank}{\blank}} (0);
      \path (0) edge  node[swap] {\smTMtransR{c}{c}}           (6);
      \path (6) edge  node[swap] {\smTMtransL{\blank}{\blank}} (7);

      \path (1) edge[loop above] node {\smAlign{\smTMtransR{a}{a}\smTMtransR{b}{b}}}                                   (1);
      \path (2) edge[loop above] node {\smTMtransR{X}{X}}                                                              (2);
      \path (3) edge[loop below] node {\smAlign{\smTMtransR{a}{a}\smTMtransR{b}{b}}}                                   (3);
      \path (4) edge[loop below] node {\smTMtransR{X}{X}}                                                              (4);
      \path (5) edge[loop right] node {\smAlign{\smTMtransL{a}{a}\smTMtransL{b}{b}\smTMtransL{c}{c}\smTMtransL{X}{X}}} (5);
      \path (6) edge[loop above] node {\smTMtransR{X}{X}}                                                              (6);

    \end{tikzpicture}
  \end{correction}
  
  \begin{question}
  \item Construire une machine de Turing déterministe qui génère le langage $\{a^n b^n a^n \mid n \ge 0\}$.
  \end{question}
  \begin{correction}
    \begin{tikzpicture}[turingMachine, baseline=(1), y=20mm]
      \state[initial]   (0) at (0,1) {$0$}; 
      \state            (1) at (1,1) {$1$}; 
      \state            (2) at (2,1) {$2$}; 
      \state[accepting] (3) at (3,1) {$3$}; 
      \state            (4) at (3,0) {$4$}; 
      \state            (5) at (2,0) {$5$}; 
      \state            (6) at (1,0) {$6$}; 

      \path (0) edge             node {\smTMtransR{\blank}{a}}                       (1);
      \path (1) edge             node {\smTMtransR{\blank}{b}}                       (2);
      \path (2) edge             node {\smTMtransL{\blank}{a}}                       (3);
      \path (3) edge[loop right] node {\smAlign{\smTMtransL{a}{a}\smTMtransL{b}{b}}} (3);
      \path (3) edge             node {\smTMtransR{\blank}{a}}                       (4);
      \path (4) edge[loop below] node {\smTMtransR{a}{a}}                            (4);
      \path (4) edge             node {\smTMtransR{b}{b}}                            (5);
      \path (5) edge[loop below] node {\smTMtransR{b}{b}}                            (5);
      \path (5) edge             node {\smTMtransR{a}{b}}                            (6);
      \path (6) edge[loop below] node {\smTMtransR{a}{a}}                            (6);
      \path (6) edge             node {\smTMtransR{\blank}{a}}                       (2);
    \end{tikzpicture}
  \end{correction}

  \begin{question}
  \item Construire une machine de Turing déterministe qui décide le langage $\{ a^{2^n} \mid n \ge 0 \}$.
  \end{question}
  \begin{correction}
    \begin{tikzpicture}[turingMachine, baseline=(0)]
      \state                (0) at (0,0) {$0$}; 
      \state[initial above] (1) at (1,0) {$1$}; 
      \state                (2) at (2,0) {$2$}; 
      \state                (3) at (3,0) {$3$}; 
      \state[accepting]     (4) at (4,0) {$4$}; 

      \path (0) edge[bend left]  node {\smTMtransR{\blank}{\blank}} (1);
      \path (1) edge[bend left]  node {\smTMtransL{\blank}{\blank}} (0);
      \path (2) edge[bend left]  node {\smTMtransR{a}{a}} (1);
      \path (1) edge[bend left]  node {\smTMtransR{a}{A}} (2);
      \path (2) edge             node {\smTMtransL{\blank}{\blank}} (3);
      \path (3) edge             node {\smTMtransR{\blank}{\blank}} (4);
      \path (0) edge[loop left]  node {\smAlign{\smTMtransL{a}{a}\smTMtransL{A}{A}}} (0);
      \path (1) edge[loop below] node {\smTMtransR{A}{A}} (1);
      \path (2) edge[loop below] node {\smTMtransR{A}{A}} (2);
      \path (3) edge[loop below] node {\smTMtransL{A}{A}} (3);
    \end{tikzpicture}
  \end{correction}
  
\end{exercice}

\endgroup
\endinput
